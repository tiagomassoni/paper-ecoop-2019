\section{Limitations and Threats to validity}
\label{sec:CaseStudyThreats}

External validity refers to generalization. Due to its size, results from the
case study cannot be generalized; its purpose is evaluating applicability and relative usefulness.
%represetantive sample
The sample is not representative, since there is no available estimate of the Javadoc-rich project
population in GitHub, then probability sample is impossible.
Our approach is as systematic as feasible in selecting the evaluated project -- manual translation
does not scale, then the sample contains only \totalSystems{} projects.
Therefore, those systems may not be representative of the real use of Javadoc in real systems; however, we were able
to detect inconsistencies between Javadoc comments and source code, as occurred in
\texttt{Utilities} class (\texttt{ABC-Music-Player} experimental unit) in which the comment for a
parameter of the methods is the right opposite of the expected behavior in the source code.

Another risk is that only 24 developers participated in the experimental study
and 142 in the survey and those samples are not representative for the community
of Java developers. Furthermore, we used only two similar data structure interfaces (queue and stack). In other domains with more complex structures, the results
may vary. In addition, the survey used only one data structure interface:
\texttt{Stack}, for asking about the comprehensibility of the interface behavior.
In other domains with more complex structures, the results can vary considerably.


Internal validity refers to causation: are changes in the dependent variable
necessarily the result of manipulations to treatments? All material for the
empirical study and the survey study is available only in English, therefore,
the experience of the Subject with English can have affected their
comprehensibility of the behavior of the provided interface.

%webprotege much bigger
Construct validity refers to correctly measuring the
dependent variable. \texttt{Dishevelled} and
\texttt{WebProt\'{e}g\'{e}} sizes set them apart from the other systems.
For instance, \texttt{Dishevelled} is more than 56 times bigger than \texttt{ABC-Music-Player}, 43
times bigger than \texttt{Jenerics}, 313 times bigger than \texttt{OOP Aufgabe3}, and 234 times
bigger than \texttt{SimpleShop}.
In order to reduce the threat on the manually-defined contracts,
all systems were annotated and reviewed by three researchers, separately.

The order in which we display the documented
interfaces on the survey form, the questions used for evaluating
comprehensibility, the kind of questions used, and the absence of opened questions
can also threat the construct validity. For dealing with these threats we
perform a pilot before applying the survey and used the results from the pilot
to improve the survey structure. In addition, the answers from developers may not be
representative of their real opinion on difficulty perception; to overcome this
threat we made a space for comments available along with the Likert-scale
questions, which are taken into account when collecting the answers.
