\section{Discussion}
\label{discussion}

We go by research question:

\subsection{RQ1. What is the success rate in implementation tasks, using three forms of public interface contracts?}
\label{rq1}




We proceed with the discussion over the research questions.

% task
\textbf{Q2.} We ask each developer to perform one task:
either implement a given interface or implement a client code for using the
methods provided by an interface -- such the use of an API (Client). 
Although developers have assigned more \emph{Very Easy} and \textit{Easy} for
the supplier task than to the client (see Figure~\ref{fig:empiricalResults}(a)),
the statistical test did not provide evidences for a significant difference
between the difficulty perceived by developers when performing the required
task.


% approach
\textbf{Q3.}  Although not supported by statistical tests since Kruskal-Wallis
rank sum test showed no difference between the approaches (p-value = 0.15), Javadoc and
\contractjdoc{} were perceived as being easier than JML (see Figure~\ref{fig:approachesEmpirical}).
This result indicates \contractjdoc{} as an approach in an intermediate level
between Javadoc and JML, with both providing runtime conformance checking. Therefore, the
proposed approach is promising: \contractjdoc{} is easy to understand
(create a code based on the comments) -- 75\% of the developers answers for
difficulty remains between \textit{Easy} and \textit{Very easy} -- and enables
the runtime checking of the comments by means of \contractjdocCompiler{}
compiler.

%code correctness
\textbf{Q4.} Concerning the code correctness, all Participants using
interfaces documented with Javadoc and \contractjdoc{} produced code
in accordance with the contracts available in the interfaces. Only one
developer using an interface with JML contracts was not able to
satisfy all the contracts: in one method the source code produced is not in
conformance with the contracts. 

% kind of task
Even though the developers have been perceived the supplier task as less
difficult than the client task, they produced code respecting the restrictions
available on the comments more times to the client task. All developers were
able to write a client code in accordance with the restrictions. Maybe the
difficulty reported by the developers is related with the attention required for
using methods provided by an interface: one needs to read the documentation
available in order to know how to use the methods; whereas implementing the
interface is more simple; mainly for the interfaces used in this experiment:
they are traditional and well-known data structures.
The results of this experiment suggest that when writing a client code,
developers tend to pay more attention to the documentation available than when
they are writing a supplier code (implementing a given interface).




\subsection{RQ2. What are the perceived understandability in using three forms of public interface contracts?}
\label{rq2}



According to the statistical tests performed, Javadoc is the most
understandable documentation approach, and \contractjdoc{} is intermediate between JML and
Javadoc, being closer to Javadoc.
This can also be seen in Figure~\ref{fig:allApproaches}.

An interesting result came from the analysis of the difficulty grouped by
experience (Figures~\ref{fig:javadocExp} to ~\ref{fig:jmlExp}): students and
professionals have perceived the same level of difficulty for JML, which is
promissing as contract-based languages are usually considered harder to be
understood by people with less experience (students, in our survey).

Overall, this survey corroborate with the results from our experiment: \contractjdoc{} is
intermediate between Javadoc and JML, being closer to Javadoc with respect to
comprehensibility.
Furthermore, the results highlight Javadoc as the easiest approach concerning the comprehensibility of
the behavior of a documented interface.


\subsection{RQ3. Can inconsistencies in Java systems be uncovered if using runtime-checkable contract expressions?}
\label{rq3}


% applying contractjdoc
For all systems (see Table~\ref{tab:Units}), we
wrote more pre- and postconditions than invariants. This result has two
explanations: first, as expected the amount of Javadoc comments over the
classes' fields in the evaluated systems is low in comparison with the amount
of Javadoc comments over method's parameters and return.

%  -- the only
% exception occurred in \texttt{SimpleShop} system because the developers have used resources from Java
% constraints for declaring some fields as not null, enabling us to establish
% invariants for them; second, even when there are no comments in a method, we are
% able to write pre- and postconditions based on analysis of method's body.

Concerning pre- and postconditions, for \texttt{ABC-Music-Player} and
\texttt{WebProt\'{e}g\'{e}} projects, we wrote almost twice as many postconditions
as preconditions.
In \texttt{ABC-Music-Player} this is related to the number of accessor methods
available and for \texttt{WebProt\'{e}g\'{e}}, the difference is related to
the available comments.
%  and to the contracts we are able to infer from method's
% body.
%
% In addition, based on the comment and the body of a method, establishing a
% postcondition appears to be simpler than a precondition, as we do not know
% the methods' clients beforehand.


We were able to detect potential inconsistencies in \texttt{ABC-Mu\-sic-Player};
the exception will be always thrown, differently from what is expected from the
commentary.
We also found a problem into
\texttt{WebProt\'{e}g\'{e}} project, in the class \texttt{OWLLiteralParser} there was one exception
in the Javadoc tag \texttt{@throws} that was not declared in the throws of the method's signature.

% conformance errors
In addition, sometimes the tests available along with the systems do
not respect the definitions from the Javadoc comments. For instance, when the
comments in natural language from \texttt{ABC-Music-Player} system are turned into
\contractjdoc{} contracts, some tests from
\texttt{MainTest}, \texttt{ParserTest}, and \texttt{SequencePlayerTest} violate
the methods' preconditions from class \texttt{Utilities}, they try to
call \texttt{Utilities}' methods by passing the value zero as the second
parameter, even though the comment declares the second parameter must be greater than zero.
This scenario also occurred in \texttt{Dishevelled} unit, the comments turned
in \contractjdoc{} contracts also enable us to detect some tests
that do not respect the restrictions available in the Javadoc comments.

As a proof of concept, \contractjdoc{} and its compiler (\contractjdocCompiler{}) enabled us to write runtime
checkable code for third-party systems based on the comments in natural
language.
As expected, the quality and variety of the contracts depended strongly on the available comments, however, we were able to
detect and correct inconsistencies and missing expressions between source code and comments.




