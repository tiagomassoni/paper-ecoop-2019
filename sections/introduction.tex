\section{Introduction}
\label{sec:introduction}

Java programmers tend to consider writing Javadoc comments as a good practice,
specially when these comments enhance public interfaces designed as third-party
libraries for client programs.
%
Despite its recognized value and practice in Java community, 
as discussed by ~\cite{liveAPI},
understanding how to use third-party libraries can be difficult.
This mainly occurs when the source of documentation is only natural language 
comments that can be incomplete and ambiguous. In addition, a well-known problem is that
documentation and implementation tend to diverge over time~\cite{Estler-etal14}; 
a Java programmer may forget to update the Javadoc documentation after performing 
an implementation change.

%
On the other hand, embedding contracts (with pre- and postconditions as executable assertions) has
long been advocated by formal methods pioneers to precisely express code behavior. 
However, only a small amount of code has such contracts~\cite{Polikarpova-etal09}. 
Part of the reason is notational, for example, in Java there is
no built-in support for contracts, besides \texttt{assert} statements.
To this end, we need an external contract framework like the Java Modeling Language (JML)~\cite{jml}
to express the full power of behavioral specifications.
%
There is evidence that programmers are more likely to use contracts in languages that support them
natively~\cite{Chalin06}, such as Microsoft's Code Contracts~\cite{codeContractsPaper} and Eiffel~\cite{eiffel}.
%
Another problem is that contracts expressed by
the existing contract languages may be useful for programmers (internal documentation), 
but it does not meet the needs of other
readers (separate/external documentation), such as third-party libraries~\cite{Leavens10,Parnas2011}. 
To use those libraries, a programmer should not need to look in the code to 
find out how to use it. 
%
Therefore, to maximize benefits, 
she must use the combination of such techniques (e.g., Javadoc and JML).

%
We propose a Javadoc-like language, called \contractjdoc{}, allowing Java programmers to add
contract specifications (pre- and postconditions), in a straightforward way, into Javadoc comments.
%
Only a few extensions are needed to allow contracts,
such as invariants.
%
To enable the power of contracts, the \contractjdoc{} compiler
translates the documented contracts into corresponding JML specifications. 
Then, these JML specifications, equipped with pre- and postconditions, are  
compiled into runtime conformance checks.
%

% how we evaluate
We evaluate our approach by performing three studies:
we first apply \contractjdoc{} to \totalSystems{} Javadoc-annotated open source
systems in order to analyse \contractjdoc{}'s applicability.
A number of previously-undetected inconsistencies between Javadoc and actual behaviour were found in some of the studied systems.
% \item %experimental study
Next, we report a study which observed 24 developers programming for Java interfaces with behaviour
documented by the conventional Javadoc, JML~\cite{jml}, and
\contractjdoc{}, within a controlled environment.
As result, developers found it easier to implement an interface with contracts
than writing a client for that interface. Also, in general pure Javadoc was straightforward to understand, but \contractjdoc{} performed better than JML, as we expected.
% \item %survey
Last, we investigate the readability of these three documentation approaches
for specifying behavior in a Java interface -- Javadoc, JML and \contractjdoc{}, by
means of a survey with 142 Java developers.
%results
Survey results did not significantly differ for \contractjdoc{} and Javadoc, which
is promising, as contracts are usually regarded as hard to read.

In summary, the main contributions of this paper are:
\begin{itemize}
\item A new approach for documenting source code -- \contractjdoc{} (Section~\ref{sec:approach});
\item A case study applying \contractjdoc{} to \totalSystems{} Javadoc-annotated open source
systems (Section~\ref{sec:caseStudy});
\item An empirical study with 24 developers programming for Java interfaces with behavior
documented by the conventional Javadoc, JML, and \contractjdoc{} (Section~\ref{sec:experiment});
\item A comprehensibility survey with 142 Java developers investigating the
readability of three documentation approaches for specifying behavior in a Java
interface: Javadoc, JML and \contractjdoc{} (Section~\ref{sec:survey}).
\end{itemize}


\subsection{Research Questions}
\label{sec:researchQuestions}

%intro paragraph
This research work investigates what is the impact of integrating contract expressions with Javadoc comments in public interfaces, in terms of applicability and usability. We first examine the impact of diverse contract documentation approaches with tasks related to the specification of data structure interfaces; as a follow-up, we inquired developers regarding understandability of contract examples. Second, we emulate the application of contract expressions to Javadoc-rich open source systems and analyzed results from runtime checking. In particular, we intend to answer the following research questions:  

\noindent\emph{RQ1. What is the success rate of public interface implementors and users in tasks, using three forms of public interface contracts?}\\
We report and discuss results from a lab study with Java developers over development tasks involving a documented data structure interface.

  
\noindent\emph{RQ2. What are the perceived difficulty in using three forms of public interface contracts?}\\
By means of a lab study with assigned tasks and a developer survey, we discuss quantitative and qualitative reports from developers regarding the forms of contract expressions that are preferred for implementation tasks.


\noindent\emph{RQ3. Can we uncover inconsistencies in Java systems if using runtime-checkable contract expressions?}\\
We collected a few open source systems based on their use of Javadoc, applied contract expressions
to each system and evaluated results in terms of detected conformance errors. Also, we discuss the problems faced when replacing Javadoc comments by contract expressions in the described context.

%TO DO: ver se é interessante colocar contribuições aqui.