\section{Introduction}
\label{sec:introduction}

%context
Java programmers tend to consider writing Javadoc comments as a good practice,
especially when these comments enhance public methods -- this case is especially crucial for API callers and implementors.
Despite its recognised value and practice in Java community~\cite{liveAPI},
understanding how to use public routines.
%problem 1
This situation mainly occurs when the source of documentation is often textual comments, potentially incomplete and ambiguous. Also, a well-known problem is that documentation and implementation tend to diverge over time~\cite{Estler-etal14}; 
a Java programmer may forget to update the Javadoc documentation after performing an implementation change. Such a scenario may produce faults related to requirements that could go unnoticed until late in the software process.

%problem 2
On the other hand, embedding contracts that follow the Design-by-Contract methodology (DbC) -- pre- and post-conditions as checkable assertions for defining a public method's behaviour, which we refer as \emph{public contracts} -- has
long been advocated by formal methods pioneers to solve issues in making programs satisfy specifications~\cite{Hoare1969,Parnas2011}. 
However, their adoption is limited~\cite{Polikarpova-etal09}. 
Part of the reason is notational, for example, in Java, there is no built-in support for public contracts.
To this end, developers might use contract frameworks like the Java Modeling Language (JML)~\cite{jml} to express the full power of behavioural specifications.

%reason 2
Another reason for their low adoption is that formal contracts, while useful for dynamic checking or static analysis, do not meet documentation needs critical to public contracts, as the case with third-party libraries~\cite{Leavens10,Parnas2011}. 
%bridge to studies
Tentative approaches do not seem to address those adoption issues~\cite{docAnalysis}.
Developers would undoubtedly benefit from the use of DbC as public contracts, although it seems undesirable to cast aside the simplicity provided by textual Javadoc specifications (Section~\ref{sec:example}).
%we did not find
To the best of our knowledge, research is needed to investigate how different contract styles in public contracts affect programming tasks.

%solution - studies
In this paper, we report results from empirical studies (Section~\ref{sec:results}) on integrating contract expressions into Javadoc comments used as public contracts. 
For this purpose, we designed and implemented a small tag-based extension to Javadoc (\contractjdoc{}) to communicate pre- and post-conditions as contract expressions.
%compilation
Its compiler translates these expressions into corresponding runtime assertions.

%studies
The first study evaluates the effectiveness and understandability~\cite{Scalabrino2017} of contract expressions within Javadoc, for both API clients and implementors, employing an experimental simulation with 24 Java developers, using three contract styles: Javadoc text, \contractjdoc{} and JML-like formal contracts. 
A follow-up judgement survey with 142 Java developers was carried out for amplifying understandability on those three styles.
Finally, we investigated anomalies in Javadoc-rich open source systems that may arise when we manually formalise textual public contracts into contract expressions, then checking conformance at runtime.

%results
The latter provided evidence on the problem of nonconformance between textual Javadoc and the implemented behaviour: we found 391 anomalies~\footnote{we refer to these issues as \emph{anomalies} because they are not necessarily bugs, but a mismatch between a specification and its implementation.} between contracts and code in one large and four small open source Java systems. Most public contracts detected are post-conditions, but often simple forms of behaviour specifications -- more elaborate public contracts would potentially result in even more anomalies (Section~\ref{sec:discussion}).

%contract styles
Our studies with contract styles, in turn, brought to our attention interesting discussion topics and hypotheses for further studies. 
%result 1
We observed the higher quality of submissions as contracts were less formal, with satisfactory results when applying \contractjdoc{} with its semi-formal approach. Also, regarding understandability, differences were discerned between different contract styles with variate levels of formality.
In general, Javadoc text was considered straightforward to understand and apply, but \contractjdoc{} expressions mixed with textual presented better results in comparison to formal contracts, although opinions about understandability and implementation outcomes seem to be unrelated (Section~\ref{sec:discussion}).

%consequences
The results are promising in establishing hypotheses to the adoption of contract languages in mainstream development, by integrating informal and formal styles of specifications for public routines.
Broad adoption of DbC in this context would bring the benefit of automatic verification to avoid mismatches between documentation and actual behaviour, as long as issues with flexibility and clarity are adequately addressed. 


\subsection{Research Questions}
\label{sec:researchQuestions}

%intro paragraph
This research work investigates the impact of integrating contract expressions with Javadoc comments in public interfaces. 
%We first examine the impact of three contract styles in API usage and implementation; as a follow-up, we inquired developers regarding understandability of examples using those contract styles. Moreover, we emulate the application of contract expressions to Javadoc-rich open source systems and analysed results from runtime checking.
In particular, we intend to answer the following research questions:  

\noindent\emph{RQ1. What is the Effect of the Contract Style on API Usage and Implementation Tasks?}\\
We report and discuss quantitative and qualitative results from an experimental simulation with Java developers over development tasks involving APIs documented with three contract styles: textual Javadoc, \contractjdoc{} (Section~\ref{sec:approach}), and JML-like formal contracts. 

\noindent\emph{RQ2. What is the Effect of the Contract Style on the Understandability of API Specifications?}\\
Using impressions given by the experimental simulation and a follow-up judgement survey, we discuss quantitative and qualitative data regarding the understandability of contract styles.

\noindent\emph{RQ3. What Kinds of Anomalies are Uncovered if Contract Expressions replace Javadoc Specifications?}\\
We collected open source systems based on their use of Javadoc and applied contract expressions
to each system, evaluating the result regarding detected anomalies (mismatches between specification and program behaviour). Also, we discuss the problems faced when replacing Javadoc comments by contract expressions in the described context.
