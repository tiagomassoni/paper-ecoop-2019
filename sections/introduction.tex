\section{Introduction}
\label{sec:introduction}

Java programmers tend to consider writing Javadoc comments as a good practice,
specially when these comments enhance public interfaces designed as third-party
libraries for client programs.
%
Despite its recognized value and practice in Java community, 
as discussed by ~\cite{liveAPI},
understanding how to use third-party libraries can be difficult.
This mainly occurs when the source of documentation is only natural language 
comments that can be incomplete and ambiguous. In addition, a well-known problem is that
documentation and implementation tend to diverge over time~\cite{Estler-etal14}; 
a Java programmer may forget to update the Javadoc documentation after performing 
an implementation change.

%
On the other hand, embedding contracts (with pre- and postconditions as executable assertions) has
long been advocated by formal methods pioneers to precisely express code behavior. 
However, only a small amount of code has such contracts~\cite{Polikarpova-etal09}. 
Part of the reason is notational, for example, in Java there is
no built-in support for contracts, besides \texttt{assert} statements.
To this end, we need an external contract framework like the Java Modeling Language (JML)~\cite{jml}
to express the full power of behavioral specifications.
%
There is evidence that programmers are more likely to use contracts in languages that support them
natively~\cite{Chalin06}, such as Microsoft's Code Contracts~\cite{codeContractsPaper} and Eiffel~\cite{eiffel}.
%
Another problem is that contracts expressed by
the existing contract languages may be useful for programmers (internal documentation), 
but it does not meet the needs of other
readers (separate/external documentation), such as third-party libraries~\cite{Leavens10,Parnas2011}. 
To use those libraries, a programmer should not need to look in the code to 
find out how to use it. 
%
Therefore, to maximize benefits, 
she must use the combination of such techniques (e.g., Javadoc and JML).

%
We propose a Javadoc-like language, called \contractjdoc{}, allowing Java programmers to add
contract specifications (pre- and postconditions), in a straightforward way, into Javadoc comments.
%
Only a few extensions are needed to allow contracts,
such as invariants.
%
To enable the power of contracts, the \contractjdoc{} compiler
translates the documented contracts into corresponding JML specifications. 
Then, these JML specifications, equipped with pre- and postconditions, are  
compiled into runtime conformance checks.
%

% how we evaluate
We evaluate our approach by performing three studies:
we first apply \contractjdoc{} to \totalSystems{} Javadoc-annotated open source
systems in order to analyse \contractjdoc{}'s applicability.
A number of previously-undetected inconsistencies between Javadoc and actual behaviour were found in some of the studied systems.
% \item %experimental study
Next, we report a study which observed 24 developers programming for Java interfaces with behaviour
documented by the conventional Javadoc, JML~\cite{jml}, and
\contractjdoc{}, within a controlled environment.
As result, developers found it easier to implement an interface with contracts
than writing a client for that interface. Also, in general pure Javadoc was straightforward to understand, but \contractjdoc{} performed better than JML, as we expected.
% \item %survey
Last, we investigate the readability of these three documentation approaches
for specifying behavior in a Java interface -- Javadoc, JML and \contractjdoc{}, by
means of a survey with 142 Java developers.
%results
Survey results did not significantly differ for \contractjdoc{} and Javadoc, which
is promising, as contracts are usually regarded as hard to read.

In summary, the main contributions of this paper are:
\begin{itemize}
\item A new approach for documenting source code -- \contractjdoc{} (Section~\ref{sec:approach});
\item A case study applying \contractjdoc{} to \totalSystems{} Javadoc-annotated open source
systems (Section~\ref{sec:caseStudy});
\item An empirical study with 24 developers programming for Java interfaces with behavior
documented by the conventional Javadoc, JML, and \contractjdoc{} (Section~\ref{sec:experiment});
\item A comprehensibility survey with 142 Java developers investigating the
readability of three documentation approaches for specifying behavior in a Java
interface: Javadoc, JML and \contractjdoc{} (Section~\ref{sec:survey}).
\end{itemize}


\subsection{Research Questions}
\label{sec:researchQuestions}

Our specifc research questions are the following:

\noindent\textbf{Q1.} Is the preciseness of \contractjdoc{}, when compared to Javadoc, useful for contract verification?\\
We collected a few open source systems based on their use of Javadoc, applied \contractjdoc{}
to each system and evaluated results in terms of detected conformance errors.

\noindent\textbf{Q2.} What is the cost of applying \contractjdoc{} into systems with Javadoc?\\
We discuss the problems faced when applying \contractjdoc{} in the described context.

\noindent\textbf{Q3.} With respect to the task performed by each developer, how
difficult is the required task?\\
We ask the developers to perform a task based on a given documented interface.
Then, we ask them to evaluate the difficulty by means of a Likert-type scale.

\noindent\textbf{Q4.} Regarding the documenting approach used by the developer,
how difficult it is for a developer to perform the required task by using the
available documented interface?\\
By using a Likert-type scale, we measure the difficulty for each approach
considered in this experiment.
  
\noindent\textbf{Q5.} Concerning the experience of the Java developers, how
difficult is it for a developer to perform the required task?\\  
By using a Likert-type scale, we measure the difficulty for each experience level
considered in this experiment.

\noindent\textbf{Q6.} In the developer's opinion, which approach, among Javadoc,
\contractjdoc{} and JML, is the most effective in communicating a routine's
behavior? \\
