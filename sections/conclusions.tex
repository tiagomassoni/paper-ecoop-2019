\section{Conclusions}
\label{sec:conclusions}

%context - tradeoffs are clear
In Java programs, textual Javadoc prevail for documenting API behaviour~\cite{docAnalysis}.
Javadoc's use is widespread, serves for important purposes -- tools for generating documentation -- but lacks, of course, automatic checking of the program against its specification.
%what we have done
The trade-off between the pragmatics of textual descriptions and formal contracts amenability to automatic analysis is explored in this paper, by three empirical studies.
%evaluation of styles
We evaluated the effect on API usage and implementation tasks of three public contract styles in Java programs: textual Javadoc, JML-like formal contracts, and a third option, a small tag-based extension to Javadoc to express pre- and post-conditions amid text and other standard tags, which we call \contractjdoc{}.
%studies
The first and second studies evaluated the effect of specification styles on the efficacy and understandability of contract expressions, either for API implementors or users, through an experimental simulation with 24 recruited participants, and a judgment survey with more than a hundred Java developers.
In the third study, we evaluated the potential of detecting anomalies by manually formalising textual Javadoc contracts from open source Java systems into contract expressions, then checking conformance at runtime.


%first question - effect on the programming task
Javadoc and \contractjdoc{} groups were more successful in submitting anomaly-free programs, while half of the formal contract group violated at least one contract clause. 
We speculate, by this outcome and by analysing the qualitative data provided by the participants, that developers, despite their awareness of the specified behaviour, were having trouble in interpreting the expressions.  
Moreover, all anomalies happened with post-conditions, which matches conclusions from previous research~\cite{Estler-etal14,sac2017} perceiving the discomfort when dealing with this kind of contract, in comparison with pre-conditions. 


%second question - effect on understandability
We also used the experimental setting for assessing understandability~\cite{Scalabrino2017} of three contract styles. 
In the answers, Javadoc contracts were regarded as the most understandable, followed by the mixed \contractjdoc{} style and, as the least understandable; nevertheless, those differences were not statistically significant. As a follow-up study, we carried out a judgement survey with 142 Java developers with at least one-year experience, in which the results were the same, this time with significant pairwise-difference effect sizes between all three styles.
This outcome is expected, as developers tend to favour informal styles for documentation~\cite{Polikarpova-etal09}. It remains to be investigated whether an intermediate approach such as \contractjdoc{} is understandable enough for fostering the use of DbC in API specifications.

%third question - can we find anomalies?
Finally, our case study applying contract expressions in open source systems resulted in finding 381 anomalies in a single Javadoc-rich system (\texttt{Dishevelled}) and ten more in other three toy systems (\texttt{ABCMusicPlayer}, \texttt{Jenerics} and \texttt{OOP Aufgabe3}), showing the potential of using contract expressions in existing Javadoc specifications.
Other interesting findings about those systems include their preference for post-conditions over pre-conditions, although most contracts were classified as \emph{common case} -- enforcing expected program properties, such as nullable variables.

% future work
%hypothesis for future work
Despite their limitations, we believe the studies presented in this paper prompt the research community to investigate some hypothesis with future enquiry. 
%h: relate perceiving understandability and correct, conforming implementation 
Making sense of the specified behaviour in contracts is undoubtedly a necessary condition to providing a program that conforms to the specification, but it is not sufficient; other forces should be considered, such as tool support, enforcing test cases or even the developer's lack of discipline.  
%h: how freedom to neglect defensive programming could be dangerous for conforming implementations 
Additionally, we could investigate whether contract conformance may be undermined by the omission of defensive programming promoted by DbC -- as it becomes pointless to add program checks to certain scenarios, other important conditions might be neglected. 


%h: how developers react to pre- and post-conditions; pre-conditions are a relief, post-conditions are neglected?
In DbC -- especially for public contracts --, pre-conditions seem easier to regard and fulfil, if compared to post-conditions. However, post-conditions are probably more fundamental to elaborate and useful contracts~\cite{Rosenblum}. Research could be carried out to test this hypothesis, and investigate the rationale behind the observations.   
%h: what are the reasons for not adopting contracts? one evidence is that they are hard to maintain in sync
Finally, the reason for the low adoption of DbC, at least for formal contracts, must be subject of further consideration. As we could observe in our case study with open-source systems, keeping the specification in sync with the program seems to be the main issue, requiring extra resources often not available in software teams with tight schedules. Automatic generation of contracts from programs~\cite{docAnalysis,atComment} seems a reasonable option to address this issue.  

%improve contractjdoc
For \contractjdoc{}, we intend to improve the tags for considering more complex contract expressions, and carry out more studies on its usability and efficiency, in particular for specifying APIs -- adaptation to other programming languages, such as C\#, would be relatively straightforward. 
For instance, we could add support to \emph{autocomplete}, applying Jaro-Winkler string distance~\cite{jaro,winkler99} as a basis.
Moreover, the overhead \contractjdoc{} adds to the instrumented code must be addressed with optimizations, perhaps using an alternative framework rather than Aspectj as foundation. 



% %the language extension
% \contractjdoc{} contracts enable runtime checking by using a language similar to the traditional Javadoc,
% and our compiler (\contractjdocCompiler{}) supports new constructs of Java language, such as the
% features from the Java 8, such as lambda expressions.
% The approach allows the use of Design by Contract~\cite{dbc} in a format closer to
% traditional Javadoc comments. 











% When evaluating \contractjdoc{} we found the
% approach almost as understandable as traditional Javadoc comments, with the
% advantage of being able to check behaviour at runtime.
% When comparing JML with \contractjdoc{}, the latter features less specification
% constructs, although results indicate programmers may feel more comfortable with
% it when writing precise behaviour for methods. 
% Besides, we found evidence
% that \contractjdoc{} is more readable than JML.
% advantages
% Furthermore, as JML contracts, contracts in \contractjdoc{} may be used in place of defensive
% programming, by specifying valid inputs for the methods and shortening the source code.

