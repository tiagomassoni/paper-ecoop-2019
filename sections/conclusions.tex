\section{Conclusions}
\label{sec:conclusions}

In this work, we present a new approach for writing comments and an
application of this approach for Java context. The approach
allows the use of Design by Contract~\cite{dbc} in a format closer to
traditional Javadoc comments. When evaluating \contractjdoc{} we found the
approach almost as understandable as traditional Javadoc comments, with the
advantage of being able to check behavior at runtime.
When comparing JML with \contractjdoc{}, the latter features less specification
constructs, although results indicate programmers may feel more comfortable with
it when writing precise behavior for methods. In addition, we found evidence
that \contractjdoc{} is more readable than JML.

% advantages
\contractjdoc{} contracts enable runtime checking by using a language similar to the traditional Javadoc,
and our compiler (\contractjdocCompiler{}) supports new constructs of Java language, such as the
features from the Java 8, such as lambda expressions.
Furthermore, as JML contracts, contracts in \contractjdoc{} may be used in place of defensive
programming, by specifying valid inputs for the methods and shortening the source code.

% future work
As future work, we plan to leverage \contractjdoc{} for
supporting autocomplete when writing the Javadoc comment. For this purpose, we
can use the Jaro-Winkler~\cite{jaro,winkler99} string distance for
autocompleting the comments.