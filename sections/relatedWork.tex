\section{Related Work}
\label{sec:relatedWork}

\textbf{Contracts.} 
As discussed, each contract-based approach choo\-ses 
a different trade-off
between expressivity/preciseness, verbosity, freedom, and tooling (e.g., runtime checking).
This is the case of JML~\cite{jml} and Microsoft's Code Contracts~\cite{codeContractsPaper}.
Both enable one to provide full behavioral specifications and their runtime checking.
Nevertheless, they lack support to allow informal specifications or to be available at third-party
library clients~\cite{Parnas2011}. They differ in the way they are written. 
The former is written as Java comments in code, whereas the latter is syntax-based 
and therefore often verbose. Without tool support to extract meaningful specifications
the contracts provided by Code Contracts are even less interesting for third-party 
libraries since they are embedded in C\# programs. Differently from these languages, the contracts expressed in \contractjdoc{} are already embedded in Javadoc comments, which are the standard approach to documenting Java programs and more likely to be available to third-party libraries.

\textbf{Javadoc Comments.}
@TCOMMENT~\cite{atComment} is an approach for testing Javadoc comments, specifically
method properties about null values and related exceptions. The approach consists of two components. The first component takes as input
source files for a Java project and automatically analyzes the English text in Javadoc comments to
infer a set of likely properties for a method in the files. The second component generates random
tests for these methods, checks the inferred properties, and reports inconsistencies. By using
\contractjdoc{}, a developer is able to write contracts richer than those for checking null values and
exceptions (as presented in Section~\ref{sec:caseStudy}).

Zhai et al.~\cite{docAnalysis} present a technique that builds models for Java API functions
by analyzing the documentation. Their models are simpler implementations in Java compared to the
original ones and hence easier to analyze. More importantly, they provide the same functionalities
as the original functions. They argue that API documentation, like Javadoc and .NET documentation,
usually contains wealthy information about the library functions, such as the behavior and exceptions they may throw. Thus it is feasible to generate models for library functions from such API documentation. In this context, the comments in \contractjdoc{} approach can be used as input for the technique in order to improve model generation.

\textbf{Testing.}
Clousot~\cite{clousot} statically checks C\#/Code
Contracts programs. The approach is based on abstract interpretation and analyzes annotated programs
to infer facts (including loop invariants), and it uses this information to discharge proof
obligations. 
AutoTest~\cite{autotest} is a collection of tools that automate the testing process for
Eiffel programs. In AutoTest, contracts are used as oracles to expected outputs for conformance checking of the programs; furthermore, AutoTest uses a randomly-guided tests generation (ARTOO~\cite{artoo}) and
supports mixing manual and automated test. 
\jmloktool{}~\cite{jmlok2} is a tool for dynamically detecting and classifying nonconformances in contract-based programs, applying randomly-generated tests (RGT) for detecting nonconformances, and a heuristics-based approach for nonconformance classification. 
These tools are for contract-based languages; in contrast, we propose and implement an approach for writing contracts in the same language as the source code, such as available in JML and Eiffel, improving documentation.

\textbf{Empirical Studies.}
There are three main related empirical studies about contract usage~\cite{typeContracts,Estler-etal14,Chalin06}.
One common conclusion about is that, in practice, developers use simple and short
contracts~\cite{typeContracts,Estler-etal14}. For instance,
\cite{typeContracts} showed that 75\% of the projects' Code Contracts are checks for the presence of data (e.g., non-null checks).
In our case study (Section~\ref{sec:caseStudy}), almost 93\% (3,711 contract
clauses out of 3,994) of the contracts we wrote remains between checks for the
presence of data and statements repeating the method's return. Chalin studied 84 Eiffel~\cite{eiffel} projects
and pointed out that developers are more likely to use contracts in languages that support them natively, like Eiffel~\cite{eiffel} or Code Contracts~\cite{codeContractsPaper}.
To support both conclusions, to write simple contracts in a native manner, \contractjdoc{} allows a Java programmer
to write those contracts as usual Javadoc comments.
