\section{Styles of Public Contracts}
\label{sec:example}

In this section, we discuss issues in specifying the behaviour of public routines. For concreteness, we provide Java examples.

\subsection{Javadoc and textual specifications}

%documentation in Javadoc - no invariant
Javadoc~\cite{javadoc-oracle} is the usual notation (and tool) for public routines (methods) in Java; it includes special tags (with symbol \texttt{@}) for structuring and pretty-printing code commentary.
The Java Platform API specification itself~\cite{java-spec}, for instance, employs Javadoc for specifying "\emph{contract[s] between callers and implementations}". In those terms, Javadoc may be a tool for applying the Design-by-Contract (DbC) methodology~\cite{dbc}, with its pre- and post-conditions around public methods, establishing the expected behaviour for each part (the contract). In the context of distributed software teams, for instance, this kind of documentation is of critical importance, because it adds preciseness to the communication between client and implementor roles. 


%java interface
Consider the bank account interface depicted in Figure~\ref{Fig-Javadoc-Bank}. For simplicity, only method {\lstinline!withdraw!} is declared. Tag \lstinline!@param! includes, for parameter \lstinline!amt!, a description that suffices as a pre-condition for \lstinline!withdraw! callers. Likewise, tags \lstinline!@return! and \lstinline!@throws! express, respectively, a normal post-condition (if it works correctly) and an exceptional post-condition (if \lstinline!TransactionException! is thrown).


\begin{figure}
\centering
\begin{lstlisting}[basicstyle=\footnotesize\ttfamily,name=figxpi]
class BankAccount {
 // ...
 /**
  * @param amt  the amount value to withdraw, where
  *             'amt' must be greater than zero 
  * @return     current 'balance' after withdraw
  * @throws     TransactionException 'balance' 
  *             remains unchanged
  *
  */
 BigDecimal withdraw(BigDecimal amt) 
   throws TransactionException {...}
 // ...
}
\end{lstlisting}
\caption{Bank Account Javadoc Specification.}
\label{Fig-Javadoc-Bank}
\end{figure}


%natural language
Contracts in such style use natural language. As a consequence, consistency between specifications and actual code behaviour cannot be automatically enforced, unless one maintains test cases in synchronicity with the Javadoc contracts. However, even test cases are hardly up-to-date with code changes~\cite{Hao2013}, so it is hard to imagine that it would be applicable to contracts. 
Furthermore, the lack of formality leads to imprecision, ambiguity, and verbosity, potentially causing program anomalies and faults.
On the other hand, using natural language does not require specialised training -- although training may be needed to communicate
ideas about program behaviour effectively -- and allows a high degree of freedom for documentation structuring. 


\subsection{Formal Contracts}


DbC is supported by construction in a few programming languages (such as Eiffel~\cite{eiffel}), or by extensions (Java Modeling Language (JML)~\cite{jml} for Java and Code Contracts~\cite{codeContractsPaper} for .NET languages) in mainstream programming languages.
For Java, JML contracts may be defined as also showed for a bank account, in Figure~\ref{Fig-JML-Bank}. Pre-conditions
are defined by the clause {\lstinline!requires!} and (normal) post-conditions by {\lstinline!ensures!}. The specification
denoted by the {\lstinline!signals!} clause
is an exceptional post-condition stating that {\lstinline!balance!} should be unchanged, when the exception \texttt{TransactionExcep\-tion} is thrown~\footnote{Usually, formal contract languages and extensions include \textit{class invariants}, which establish conditions that must remain true between executions of a given class. In this paper, we do not focus on invariants since their use as public contracts is rare, as it would require public or abstract fields.}.

\begin{figure}
\begin{lstlisting}[basicstyle=\footnotesize\ttfamily,name=figxpi]
class BankAccount {
 BigDecimal balance;

 //@ requires amt.compareTo(new BigDecimal("0") > 0 
 //@ && amt.compareTo(balance) <= 0;
 //@ ensures balance.equals(\old(balance.subtract(amt)));
 //@ ensures \result.equals(balance);
 //@ signals (TransactionException) 
 //@   balance.equals(\old(balance));
 BigDecimal withdraw(BigDecimal amt) 
   throws TransactionException {...}
 // ...
}
\end{lstlisting}
\caption{The JML specifications for the bank account.}
\label{Fig-JML-Bank}
\end{figure}

In this style, formal contracts precisely describe what must be true when the method is called, what must be true when the method return or when it returns abnormally. A critical property of such contracts is that they are machine-checkable, either by assertion testing or static analysis~\cite{Chalin06}.
% Also for object-oriented languages, contracts can describe object invariants that must hold for an object in all of its visible states -- in this paper,.
Nevertheless, using JML-like formal contracts might require some level of training, becoming, to some extent, hard to read or write, and thus is often used sparingly~\cite{Chalin06,Polikarpova-etal09,typeContracts}.
Besides, if the code in Figure~\ref{Fig-JML-Bank} were an interface, with no available code, the specification would be required. However, usually formal contracts are only available for the interface implementors, or not available at all~\cite{Parnas2011}.

%The Documentation Dilemma
Therefore, we face a dilemma concerning program documentation. If we use formal contracts, the result is more precise documentation, with the possibility of automatic checks.
Contrarily, by using an informal documentation approach such as Javadoc, we face the lack of precision and potential ambiguity, despite its flexibility and simplicity. This trade-off leads us to the following inquiries: is it possible to have the best of both worlds, mixing informal documentation and contract specification within a unified framework? In this case, what would be the effect of using such an approach to API clients and implementors? This paper improve evidence on those questions.

%In the following, we discuss how \contractjdoc{} provides means to combine the benefits
% and overcome the main limitations 
%of the existing documentation approaches discussed previously.