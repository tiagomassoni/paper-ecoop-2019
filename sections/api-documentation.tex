\section{Background on Documentation Approaches}
\label{sec:example}

In this section, we discuss the existing problems and benefits in documenting programs. For concreteness,
we exemplify documentation approaches for the Java language.

\subsection{A Running Example}

Figures~\ref{Fig-JML-Bank} and~\ref{Fig-Javadoc-Bank} illustrate
a simple bank account class, documented with JML and Javadoc respectively. For simplicity, we consider only the {\lstinline!withdraw!} method.

In the JML specifications (Figure~\ref{Fig-JML-Bank}), preconditions
are defined by the clause {\lstinline!requires!} and (normal) postconditions by {\lstinline!ensures!}. The specification
denoted by the {\lstinline!signals!} clause
is an exceptional postcondition, which says that
the {\lstinline!balance!} should be unchanged, when the exception
\texttt{TransactionExcep\-tion} is thrown.
The invariant defined in class {\lstinline!BankAccount!} restricts the account's balance
to be always greater than or equal to zero.
In the Javadoc side (Figure~\ref{Fig-Javadoc-Bank}), the precondition is
informally written with the tag \lstinline!@param!, normal postcondition with
\lstinline!@return!, and exceptional postcondition with \lstinline!@throws!.
And the invariant within a Javadoc block comment
above \lstinline!balance!'s declaration.

\begin{figure*}
\centering
\begin{subfigure}{.48\textwidth}
\begin{lstlisting}[basicstyle=\footnotesize\ttfamily,name=figxpi]
class BankAccount {
 double balance;
 //@ invariant balance >= 0;

 //@ requires amt > 0 && amt <= balance;
 //@ ensures balance == \old(balance - amt);
 //@ ensures \result == balance;
 //@ signals (TransactionException) 
 //@   balance == \old(balance);
 double withdraw(double amt) 
   throws TransactionException {...}
 // ...
}
\end{lstlisting}
\caption{The JML specifications for the bank account.}
\label{Fig-JML-Bank}
\end{subfigure}%
\begin{subfigure}{.48\textwidth}
\begin{lstlisting}[basicstyle=\footnotesize\ttfamily,name=figxpi]
class BankAccount {
 /**
  * The overall balance should be
  * greater than or equal to zero
  */
 double balance;

 /**
  * @param amt  the amount value to withdraw, where
  *             'amt' should be greater than zero and
  *             less than or equal to 'balance'
  * @return     current 'balance' after withdraw
  * @throws     TransactionException 'balance' 
  *             remains unchanged
  *
  */
 double withdraw(double amt) 
   throws TransactionException {...}
 // ...
}
\end{lstlisting}
\caption{The Javadoc documentation for the bank account.}
\label{Fig-Javadoc-Bank}
\end{subfigure}
\caption{JML and Javadoc documentation for the bank account.}
\label{fig:test}
\end{figure*}


\subsection{Natural Language Approach}

Often natural language documentation
is found in the source code, as in Javadocs
(see Figure~\ref{Fig-Javadoc-Bank}).
Consistency between documentation and code cannot be automatically enforced, as there is
no formal connection between the comments and the code.
The lack of formality leads to imprecision, ambiguity, and verbosity (to explain the constraints).
This poses translation problems to use natural language description for automated tools, such as in testing or debugging.

On the other hand, natural language documentation does have its advantages.
It does not require special training -- although training may be needed to effectively communicate
ideas about programs in a natural language, such as English -- and allows a high
of freedom to a programmer structure the documentation. Furthermore, client's programs may benefit from the available documentation about the program usage.


\subsection{Specification/Contract Language Approach}

Contracts are a popular tool for specifying the functional behavior of software.
A method's contracts, unlike natural approaches, precisely and unambiguously describe
what must be true when the method is called (precondition), what must be true when the method
return (normal postcondition) or when it returns abnormally (exceptional postcondition).
In addition for object-oriented languages, contracts can describe object invariants that must
hold for an object in all of its visible states.

Nevertheless, a formal specification language, like JML, can require some level of training,
hence becoming, to some extent, hard to read and write, and hence is often used
sparingly~\cite{Chalin06,Polikarpova-etal09,typeContracts}.
In addition, assuming the code in Figure~\ref{Fig-JML-Bank} as within a library, -- where the source code
is unavailable -- we need the specification to reason about the library's usage. But, usually, the
specifications like the ones in Figure~\ref{Fig-JML-Bank} are only available for the implementor, thus
becoming not useful for its clients~\cite{Parnas2011}.


\subsection{The Documentation Dilemma}

It is clear that we face a dilemma with respect to program documentation.
If we use JML to provide formal documentation for contracts, the result is a
more precise documentation, with the possibility of automatic checks. However,
also results in a less flexible documentation in terms of using natural language to structure it.
In addition, formal specifications are more interesting for the implementor/maintainer of the Java program
(Section~\ref{sec:experiment}).

If we go back to a more flexible and informal documentation approach such as Javadoc, we face the
lack of precision and potential ambiguity, even though Javadoc comments are more useful for third-party
libraries users. This dilemma leads us to the following research question: Is it possible to have
the best of both worlds? That is, can we mix informal documentation and contract specification within a unified approach?

In the following, we discuss how \contractjdoc{} provides means to combine the benefits
% and overcome the main limitations 
of the existing documentation approaches discussed previously.