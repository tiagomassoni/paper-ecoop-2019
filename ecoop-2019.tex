\documentclass[a4paper,UKenglish]{lipics-v2018}
%This is a template for producing LIPIcs articles. 
%See lipics-manual.pdf for further information.
%for A4 paper format use option "a4paper", for US-letter use option "letterpaper"
%for british hyphenation rules use option "UKenglish", for american hyphenation rules use option "USenglish"
% for section-numbered lemmas etc., use "numberwithinsect"

\usepackage{listings}
\usepackage{color}
\usepackage{hyperref}
\usepackage{pdflscape}
\usepackage{enumerate}
\usepackage{hyperref}
\usepackage{graphicx}
\usepackage{lmodern}
\usepackage{url}
\usepackage{multirow}

%% Some recommended packages.
\usepackage{booktabs}   %% For formal tables:
                        %% http://ctan.org/pkg/booktabs
\usepackage{subcaption} %% For complex figures with subfigures/subcaptions
                        %% http://ctan.org/pkg/subcaption

\definecolor{mygreen}{rgb}{0.0, 0.5, 0.0}
\definecolor{myred}{rgb}{0.5,0.0,0.2}
\definecolor{LightGray}{gray}{0.9}

%*************************************************
\input{jml-listings}
%\lstset{basicstyle=\ttfamily, keywordstyle=\bfseries, stringstyle=\ttfamily, mathescape=true, language={[JML]Java}}

\lstset{showspaces=false,
  showtabs=false,
  commentstyle=\color{mygreen}\bfseries,
  keywordstyle=\color{myred}\bfseries,
  stringstyle=\color{blue},
  language=Java,
  basicstyle=\ttfamily,
  showstringspaces=false,
}

\definecolor{darkgrey}{rgb}{0.70, 0.70, 0.70}
\definecolor{lightgrey}{rgb}{0.80, 0.80, 0.80}
\newcommand{\shd}[1]{\colorbox{lightgrey}{#1}}
\newcommand{\shdk}[1]{\colorbox{darkgrey}{#1}}
\newcommand{\jmloktool}[1]{\textsc{JmlOk2}}
\newcommand{\contractjdoc}[1]{\textsc{ContractJDoc}}
\newcommand{\contractjdocCompiler}[1]{\textsc{ajmlc-contractjdoc}}

%commands for constant values
\newcommand{\totalClauses}[1]{3,993}
\newcommand{\totalPre}[1]{1,952}
\newcommand{\totalPost}[1]{2,029}
\newcommand{\totalInv}[1]{19}
\newcommand{\totalCode}[1]{190,655}
\newcommand{\totalSystems}[1]{six}

% respondants of our survey
\newcommand{\surveyRespondants}[1]{142}

\usepackage{microtype}%if unwanted, comment out or use option "draft"
\bibliographystyle{plainurl}

\title{Semi-formal Javadoc for API Documentation: Trade-offs between Dynamic Checking and Comprehension}

\titlerunning{Semi-formal Javadoc for API Documentation}%optional, please use if title is longer than one line

\author{Alysson Milanez}{Department of Systems and Computing, UFCG, Brazil}{alyssonfilgueira@copin.ufcg.edu.br}{}{}

\author{Tiago Massoni}{Department of Systems and Computing, UFCG, Brazil}{massoni@dsc.ufcg.edu.br}{}{}

\author{Henrique Reb\^{e}lo}{Informatics Center, UFPE, Brazil}{hemr@cin.ufpe.br}{}{}

\author{Rohit Gheyi}{Department of Systems and Computing, UFCG, Brazil}{rohit@dsc.ufcg.edu.br}{}{}

\author{Gary Leavens}{Department of Computer Science, UCF, USA}{leavens@cs.ucf.edu}{}{}

\authorrunning{A., Milanez et al.}

\Copyright{Alysson Milanez and Tiago Massoni and Henrique Reb\^{e}lo and Rohit Gheyi and Gary Leavens}%mandatory, please use full first names. LIPIcs license is "CC-BY";  http://creativecommons.org/licenses/by/3.0/


\subjclass{
\ccsdesc[500]{Software and its engineering~Software verification and validation, Software and its engineering~Domain specific languages}
}% mandatory: Please choose ACM 2012 classifications from https://www.acm.org/publications/class-2012 or https://dl.acm.org/ccs/ccs_flat.cfm . E.g., cite as "General and reference $\rightarrow$ General literature" or \ccsdesc[100]{General and reference~General literature}. 

\keywords{design-by-contract, documentation, runtime checking, Javadoc}

%\category{}%optional, e.g. invited paper
%\relatedversion{}%optional, e.g. full version hosted on arXiv, HAL, or other respository/website
\supplement{TBD}%optional, e.g. related research data, source code, ... hosted on a repository like zenodo, figshare, GitHub, ...
%\funding{}%optional, to capture a funding statement, which applies to all authors. Please enter author specific funding statements as fifth argument of the \author macro.

\acknowledgements{I want to thank \dots}%optional

\nolinenumbers %uncomment to disable line numbering
%\hideLIPIcs  %uncomment to remove references to LIPIcs series (logo, DOI, ...), e.g. when preparing a pre-final version to be uploaded to arXiv or another public repository

\begin{document}


\maketitle

\begin{abstract}
Doc comments counts as best practice for API documentation, but does not compel
programmers to provide a precise or complete behaviour of a routine.
%
This is, on the other hand, the very essence of contracts, but programmers are resistant to use contracts in their modules.
%, as it does not meet the need of usual readers, such as third-party library users.
%
In this scenario, we propose
% a Javadoc-like language for integrating contracts,
an approach for documenting code by means of contracts into brackets associated
with tag comments, called \contractjdoc{}, and apply the approach to the Java context.
%In support of the research contributions, 
This paper presents three empirical studies:
(1) we manually applied \contractjdoc{} to two real Javadoc-rich open source
systems and four toy projects, in which we detected inconsistencies between
the existing Javadoc and the corresponding source code; (2) an empirical study
with 24 programmers using three different documenting approaches, including
\contractjdoc{}, whose results did not significantly differ, and (3) we surveyed 142 Java programmers regarding \contractjdoc{}'s
readability in contrast to other two approaches (Javadoc and JML), in which
results did not significantly differ for \contractjdoc{} and Javadoc, which is
promising, since contracts are regarded as hard to read.
% conclusions?
Therefore, \contractjdoc{} is an approach that can foster contracts adoption,
helping to improve software quality.
\end{abstract}

\section{Introduction}
\label{sec:introduction}

%context
Java programmers tend to consider writing Javadoc comments~\cite{javadoc-oracle} as a good practice,
especially when these comments enhance public methods -- this case is especially crucial for API callers and implementors. Despite its recognised value and practice in Java community, understanding how to use public routines using contracts is widely ignored~\cite{liveAPI}.
%problem 1
This situation occurs because the source of documentation is textual comments, which are potentially incomplete and ambiguous. Also, a well-known problem is that documentation and implementation tend to diverge over time~\cite{Estler-etal14}; 
a developer may forget to update the Javadoc documentation after performing an implementation change. Such a scenario may produce faults related to requirements that could go unnoticed until late in the software process.

%problem 2
On the other hand, embedding contracts that follow the Design-by-Contract methodology (DbC) -- pre- and post-conditions as checkable assertions for defining a public method's behaviour, which we refer as \emph{public contracts} -- has
long been advocated by formal methods pioneers for program correctness~\cite{Hoare1969,Parnas2011}. 
However, their adoption is limited~\cite{Polikarpova-etal09}. 
Part of the reason is notational, for example, in Java, there is no built-in support for public contracts.
To this end, developers might use contract frameworks like the \emph{Java Modeling Language} (JML)~\cite{jml} to express the full power of behavioural specifications.

%reason 2
Another reason for their low adoption is that formal contracts, while useful for dynamic checking or static analysis, do not meet documentation needs that are critical to public routines, as the case with third-party libraries~\cite{Leavens10,Parnas2011}.
APIs usually do not provide their source code, so clients can only infer the expected behaviour by reading the public documentation, and formal contracts, in this scenario, are seldom applied.
%bridge to studies
Tentative approaches do not seem to address those adoption issues~\cite{docAnalysis}.
Developers would undoubtedly benefit from the use of DbC as public contracts, although it seems undesirable discard the simplicity provided by textual Javadoc specifications (Section~\ref{sec:example}).
%we did not find
Research is needed to investigate how different contract styles in public contracts affect programming tasks.

%solution - studies
In this paper, we report results from empirical studies (Section~\ref{sec:results}) on integrating contract expressions into Javadoc comments used as public contracts. 
For this purpose, we designed and implemented a small tag-based extension to Javadoc (\contractjdoc{}) to communicate pre- and post-conditions as contract expressions.
%compilation
Its compiler translates these expressions into corresponding runtime assertions.

%studies
The first study evaluates the \emph{effectiveness and understandability}~\cite{Scalabrino2017} of contract expressions within Javadoc, for both API clients and implementors, employing an experimental simulation with 24 Java developers, using three contract styles: Javadoc text, \contractjdoc{} and JML-like formal contracts. 
A follow-up judgement survey with 142 Java developers was carried out for amplifying the enquiry on understandability for the contract styles. Finally, we investigated anomalies~\footnote{We refer to these issues as \emph{anomalies} because they are not necessarily bugs, but a mismatch between a specification and its implementation.} in Javadoc-rich open source systems that may arise when we manually formalise textual public contracts into contract expressions, before checking conformance at runtime.

%results
The case study provided evidence on the problem of nonconformance between textual Javadoc and the implemented behaviour: we found 391 anomalies between contracts and code in one large and four small open source Java systems. 
The detected anomalies reflect potential errors already present in the code or mismatches between the code and the contracts (i.e., possible errors in the contracts).
Most public contracts detected are post-conditions, but often simple forms of behaviour specifications -- more elaborate public contracts would potentially result in even more anomalies, because either more precise specifications would reveal more anomalies or longer specifications would be problematic themselves (Section~\ref{sec:discussion}).

%contract styles
Our studies with contract styles, in turn, brought to our attention interesting discussion topics and hypotheses for further studies. 
%result 1
We observed the higher quality of submissions as contracts were less formal, with satisfactory results when applying \contractjdoc{} with its semi-formal approach. Also, regarding understandability, differences were discerned between different contract styles with different levels of formality.
In general, Javadoc text was considered straightforward to understand and apply, but \contractjdoc{} expressions mixed with textual presented better results in comparison to formal contracts (in the long tradition of Z specifications~\cite{zed}), although opinions about understandability, and implementation outcomes, seem to be unrelated (Section~\ref{sec:discussion}).

%consequences
The results are promising in establishing hypotheses to the adoption of contract languages in mainstream development, by integrating informal and formal styles of specifications for public routines.
Broad adoption of DbC in this context would bring the benefit of automatic verification to avoid mismatches between documentation and actual behaviour, as long as issues with flexibility and clarity are adequately addressed. 


\subsection{Research Questions}
\label{sec:researchQuestions}

%intro paragraph
This research work investigates the impact of integrating contract expressions with Javadoc comments in public interfaces. 
%We first examine the impact of three contract styles in API usage and implementation; as a follow-up, we inquired developers regarding understandability of examples using those contract styles. Moreover, we emulate the application of contract expressions to Javadoc-rich open source systems and analysed results from runtime checking.
In particular, we intend to answer the following research questions:  

\noindent\emph{RQ1. What is the Effect of the Contract Style on API Usage and Implementation Tasks?}\\
We report and discuss quantitative and qualitative results from an experimental simulation with Java developers over development tasks involving APIs documented with three contract styles: textual Javadoc, \contractjdoc{} (Section~\ref{sec:approach}), and JML-like formal contracts. 

\noindent\emph{RQ2. What is the Effect of the Contract Style on the Understandability of API Specifications?}\\
Using impressions given by the experimental simulation and a follow-up judgement survey, we discuss quantitative and qualitative data regarding the understandability of contract styles. 
Our approach to \emph{understandability} is to evaluate the comprehension of functionalities provided by a given routine, in terms of the expected inputs and outputs; however, it is a composite construct, whose measurement may be limited~\cite{4019969,Scalabrino2017}.


\noindent\emph{RQ3. What Kinds of Anomalies are Uncovered if Contract Expressions replace Javadoc Specifications?}\\
We collected open source systems based on their use of Javadoc and applied contract expressions
to each system, evaluating the result regarding detected anomalies (mismatches between specification and program behaviour). Also, we discuss the problems faced when replacing Javadoc comments by contract expressions in the described context.

\section{Styles of Public Contracts}
\label{sec:example}

In this section, we discuss issues in specifying the behaviour of API interfaces. For concreteness, we provide Java examples.

\subsection{Javadoc and textual specifications}

%documentation in Javadoc - no invariant
Javadoc~\cite{javadoc-oracle} is the usual notation (and tool) for API specifications in Java; it includes special tags (with symbol \@) for structuring and pretty-printing code commentary.
The Java Platform API specification itself~\cite{java-spec}, for instance, employs Javadoc for specifying "contract[s] between callers and implementations."
In those terms, Javadoc may be a tool for applying the Design-by-Contract (DBC) methodology~\cite{dbc}, with its pre- and post-conditions around public methods, establishing the expected behaviour for each part (the contract). In the context of distributed software teams, for instance, this kind of documentation is of critical importance. 


%java interface
Consider the bank account API depicted in Figure~\ref{Fig-Javadoc-Bank}\footnote{we consider API the public members of a Java class, or a Java interface}. For simplicity, only method {\lstinline!withdraw!} is declared. Tag \lstinline!@param! includes, for parameter \lstinline!amt!, a description that suffices as a pre-condition for \lstinline!withdraw! callers. Likewise, tags \lstinline!@return! and \lstinline!@throws! document, respectively, a normal post-condition (if it works correctly) and a exceptional post-condition (if \lstinline!TransactionException! is thrown).


\begin{figure}
\centering
\begin{lstlisting}[basicstyle=\footnotesize\ttfamily,name=figxpi]
class BankAccount {
 // ...
 /**
  * @param amt  the amount value to withdraw, where
  *             'amt' must be greater than zero 
  * @return     current 'balance' after withdraw
  * @throws     TransactionException 'balance' 
  *             remains unchanged
  *
  */
 double withdraw(double amt) 
   throws TransactionException {...}
 // ...
}
\end{lstlisting}
\caption{Bank Account API Specification using Javadoc.}
\label{Fig-Javadoc-Bank}
\end{figure}


%natural langague
Contracts in such style use natural language. As a consequence, consistency between specifications and actual code behaviour cannot be automatically enforced, unless one maintains test cases in synchronicity with the Javadoc contracts. However, even test cases are hardly up-to-date to code changes~\cite{Hao2013}, so it is hard to imagine that would be applicable. 
Furthermore, the lack of formality leads to imprecision, ambiguity, and verbosity, potentially leading to program anomalies and faults.
This poses translation problems to use natural language description for automated tools, such as in testing or debugging.

On the other hand, using natural language does not require special training -- although training may be needed to effectively communicate
ideas about program behaviour -- and allows a high degree of freedom for documentation structuring. 


\subsection{Formal Contracts}


DBC is supported by contruction in a few programming languages (such as Eiffel~\cite{eiffel}), or by extensions (Java Modeling Language (JML)~\cite{jml} for Java and Code Contracts~\cite{codeContractsPaper} for .NET languages) in mainstream programming languages.
For Java, JML contracts may be defined as showed in the same API for a bank account, in Figure~\ref{Fig-JML-Bank}. Pre-conditions
are defined by the clause {\lstinline!requires!} and (normal) post-conditions by {\lstinline!ensures!}. The specification
denoted by the {\lstinline!signals!} clause
is an exceptional post-condition stating that {\lstinline!balance!} should be unchanged, when the exception \texttt{TransactionExcep\-tion} is thrown.

\begin{figure}
\begin{lstlisting}[basicstyle=\footnotesize\ttfamily,name=figxpi]
class BankAccount {
 double balance;

 //@ requires amt > 0 && amt <= balance;
 //@ ensures balance == \old(balance - amt);
 //@ ensures \result == balance;
 //@ signals (TransactionException) 
 //@   balance == \old(balance);
 double withdraw(double amt) 
   throws TransactionException {...}
 // ...
}
\end{lstlisting}
\caption{The JML specifications for the bank account.}
\label{Fig-JML-Bank}
\end{figure}

In this style, formal contracts for methods, unlike natural approaches, precisely describe what must be true when the method is called, what must be true when the method return or when it returns abnormally. A critical property of such contracts is that they are machine-checkable, either by assertion testing or static analysis\~cite{Chalin06}.
%In addition for object-oriented languages, contracts can describe object invariants that must hold for an object in all of its visible states -- in this paper,.
Nevertheless, using JML-like formal contracts might require some level of training, hence becoming, to some extent, hard to read and write, and hence is often used sparingly~\cite{Chalin06,Polikarpova-etal09,typeContracts}.
In addition, assuming the code in Figure~\ref{Fig-JML-Bank} is an API with no available code, the specification is needed to use the API. However, usually formal contracts are only available for the API implementors, thus becoming not useful for its clients~\cite{Parnas2011}.


%The Documentation Dilemma
It is clear that we face a dilemma with respect to program documentation. If we use JML to provide formal documentation for contracts, the result is a more precise documentation, with the possibility of automatic checks. However, also results in a less flexible documentation in terms of using natural language to structure it.
If we refer to a more flexible and informal documentation approach such as Javadoc, we face the lack of precision and potential ambiguity, even though Javadoc comments are more useful for third-party libraries users. This dilemma leads us to the following  inquiries: is it possible to have the best of both worlds, mixing informal documentation and contract specification within a unified framework? In this case, what would be the effect of using such approach to API usage and development? This paper tries to enhance evidence on DBC development and languages by focusing on those questions.

%In the following, we discuss how \contractjdoc{} provides means to combine the benefits
% and overcome the main limitations 
%of the existing documentation approaches discussed previously.
\section{A Javadoc Extension for Public Contracts}
\label{sec:approach}

%update to be more general
In order to perform studies on approaches for specifying public contracts, we propose \contractjdoc{}, a simple extension to the Javadoc-tagging systems with contracts. Its compilation system is built on the top of the AspectJML compiler~\cite{aspectjml}, providing support for runtime checking of the contracts.
\contractjdoc{} allows programmers to document \emph{contract expressions} amid Javadoc header for methods. With just a few new tags, in addition to the standard Javadoc tags, one can write contracts as Javadoc comments that are compiled to runtime checkable code. 
%The \contractjdoc{} approach fulfills the gap between informal documentation (as with Javadoc) and formal specification (such as JML~\cite{jml} or Code Contracts~\cite{codeContractsPaper}).

In the remaining of this section, we present, based on a running example, how \contractjdoc{} supports a mixed approach, which combines textual documentation with a limited set of formal features from the examples in Section~\ref{sec:example}.

\subsection{Language Extension Design}

The \contractjdoc{} tags act as traditional Javadoc tags,  embedded within block comments. 
The main idea is to allow a mix between the traditional Javadoc syntax and JML-like notation; 
JML seems apropriate in this context, since it is built over Java expressions, with exception of a few logical operators (such as \texttt{==>} -- implication).
Embedding contracts allows for contract expressions (e.g., pre-conditions) in the existing Javadoc comments
and making them machine discoverable through the use of
marker brackets within those comments.
%semantic-neutral
In this proposal, we consider the textual comments surrounding the formal expressions semantically neutral -- we assume the expression does not change its semantics. 

%benefits
The potential benefit of embedding contract expressions in a more natural setting for the developer is his/her ability to remain within a single artifact which is purpose-built for writing API specifications. 
This is specially true because the overwhelming majority of contracts that programmers write in
practice are short and simple~\cite{Estler-etal14,typeContracts}. For instance, in 75\% of Code Contracts~\cite{codeContractsPaper} projects, the written contracts are basic checks for the presence of data (e.g., non-null checks)~\cite{typeContracts}. In such scenarios, there is no additional effort in embedding such contracts in Javadoc comments using our \contractjdoc{} approach.

\subsubsection{Pre-conditions}
%Recall the precondition illustrated in Section~\ref{sec:example}. We discussed two ways to document such a precondition, a formal one in JML (see Figure~\ref{Fig-JML-Bank}) and an informal one with plain Javadoc comments (see Figure~\ref{Fig-Javadoc-Bank}).
In \contractjdoc, the pre-condition for method \lstinline!withdraw! from Section~\ref{sec:example} can be rewritten as the excerpt in Figure~\ref{fig:pre-example}.

\begin{figure}
\centering
\begin{lstlisting}[basicstyle=\footnotesize\ttfamily,name=figxpi, frame=lines, mathescape=true]
 /**
  * @param amt the amount value to withdraw, 
  *   where $\shd{[amt > 0 \&\& amt <= balance]}$
  */
 double withdraw(double amt) 
   throws TransactionException {...}
\end{lstlisting}
\caption{\texttt{withdraw}'s pre-condition.}
\label{fig:pre-example}
\end{figure}

%explain
Tag \lstinline!@param! documents parameter \lstinline!amt! of \lstinline!double! type.
Besides the usual comments, we have added a boolean expression
surrounded by brackets; these brackets indicate assertions internally to the \contractjdoc{} compiler, so the comments can be turned into an executable precondition checking.
An alternative (not showed) is to replace tag \lstinline!@param! by \lstinline!@requires! or \lstinline!@pre!. Both can be used to document a precondition constraint; the main difference is that they are not part of the standard Javadoc tagging system.

\subsubsection{Post-conditions}

We may use \contractjdoc{} for post-conditions as the example in Figure~\ref{fig:post-example}.


\begin{figure}
\centering
\begin{lstlisting}[basicstyle=\footnotesize\ttfamily,name=figxpi, frame=lines, mathescape=true]
 /**
  * @return amt the current balance after withdraw,
  *   that is $\shd{[@return == balance]}$
  * @throws TransactionException the 'balance' does
  *  not change, that is $\shd{[balance == @old(balance)]}$
  */
 double withdraw(double amt) 
   throws TransactionException {...}
\end{lstlisting}
\caption{\texttt{withdraw}'s post-condition.}
\label{fig:post-example}
\end{figure}

Tags \lstinline!@return! and \lstinline!@throws! document normal and exceptional post-conditions, respectively, with their respective expressions
expressed within brackets.
%Tag \lstinline!@return! appears again within the brackets. This inner is allowed (in \contractjdoc{}) and allows one
%to use the value of the method's  returns to write the contract regarding
%the normal postcondition. A similar tag, \lstinline!@result! derived from the JML syntax,
%can also be employed instead of the inner use of the \lstinline!@return! tag.
Tag \lstinline!@old! refers to expressions or fields in their pre-state, used only in post-conditions.

As with preconditions, \contractjdoc{} offers three tags for expressing postconditions.
For normal postconditions, similar JML-based tags \lstinline!@ensures! and \lstinline!@post! may be used instead of \lstinline!@return!.
For \lstinline!@throws! tag, the standard Javadoc offers a surrogate tag \lstinline!@exception!. 
Derived from JML, we can also employ \lstinline!@signals! to document and constrain exceptional behavior.

% \subsubsection{Documenting Invariants}

% Beyond the support for pre- and postconditions, \contractjdoc{} make the use of invariants also
% available by means of \texttt{@inv} tags.
% The format of writing is the same as those
% for pre- and postconditions.
% The difference is related to the semantics: while pre- and postconditions apply to a specific
% method, an invariant applies for all methods from a class. For invariants, we follow the
% semantics of the JML language. For more information, please refer
% to~\cite{jml}.
% The invariant contract, described in Section~\ref{sec:example}, can be written
% as follows:
% \begin{lstlisting}[basicstyle=\footnotesize\ttfamily,name=figxpi, frame=lines, mathescape=true]
% class BankAccount {
%  /**
%   * @inv The overall balance should be $\shd{[balance >= 0]}$
%   */
%  double balance;
%  //...
% }
% \end{lstlisting}
% Invariant declarations may be placed above the field
% declarations (as in the example), or  above the
% class declaration as a valid Javadoc block comment.

\subsection{Supporting Infrastructure}

The \contractjdoc{} compiler is based on the open source AspectJML/ajmlc compiler~\cite{aspectjml,ajmlc,Rebelo-etal08}.
Unlike the standard JML compiler jmlc~\cite{jmlc-compiler}, ajmlc presents code optimizations and improved error reporting~\cite{ajmlc}.
Also, AspectJML enables the modularization
of crosscutting contracts that can arise in standard
JML specifications~\cite{aspectjml}.

We adapted the front-end of the AspectJML/ajmlc compiler to convert/preprocess the \contractjdoc{} tags into the corresponding JML features, like pre- and post-conditions.
After conversion, the compilation occurs as usual and generates aspects to runtime checking the
contracts. See Figure~\ref{fig:compilerInfra}
for an overview of the compilation strategy.
First, source code with \contractjdoc{}
contract expressions goes through a tag processor and a type checker. Next, a runtime assertion aspect is generated, which is woven to the source code by the aspectj compiler, producing bytecode with assertions, amenable to runtime checking.


\begin{figure}[h]
\centering
\includegraphics[width=1.0\textwidth]{figs/compilerInfra}
\caption{Compilation Infrastructure for ContractjDoc.}
\label{fig:compilerInfra}
\end{figure}


\section{Methodology}
\label{sec:researchDesign}

In this section, we explain the research design of each study performed for
evaluating the proposed approach.

\subsection{Case Study}
\label{sec:caseStudy}
%goal
This study aims at assessing \contractjdoc{} applicability, with respect to automation benefits, from the point of view of Java developers. 
%metric
We observe the results from applying \contractjdoc{} to \totalSystems{} real,
Javadoc-rich open source systems; all their method-level
Javadoc annotations are manually translated to \contractjdoc{}, before
running tests looking for mismatches between specifications and actual method
behavior.

\subsubsection{Systems Selection} 
\label{sec:systems}

%which systems
The case study was performed on a convenience sample: \totalSystems{}
Javadoc-rich open source systems available at GitHub\footnote{\url{https://github.com/}} repository.
%criteria
They were selected based on the presence of method-level Javadoc annotations. 
Projects are searched by the following set of key phrases: ``must be'', ``must not be'', ``should
be'', ``should not be'', ``greater than'', ``not be null'', ``less than'' into Javadoc
comments.
After some visual filtering, we collected the five most important classes in
each system, based on overall dependence, and check whether those classes
contained method-level Javadoc comments for most of their methods. If so, the
system is selected. Finally, we checked whether the system presented a suite of
unit test, which are run during the case study to detect inconsistencies. We
were able to find four systems meeting these criteria, although we performed the
manual translation to six systems.

%system descriptions
While \texttt{ABC-Music-Player}\footnote{\url{https://github.com/deepakn94/ABC-Music-Player}}
plays music from an ABC file (part of a project assignment from MIT class
6.005), \texttt{Dishevelled}\footnote{\url{https://github.com/heuermh/dishevelled}} hosts
free and Open Source libraries for several user interface components and
supporting code, with emphasis on views and editors for complex data structures, like collections, sets, lists, maps, graphs, and
matrices; \texttt{Jenerics}\footnote{\url{https://github.com/mriedel/Jenerics}} is a general-purpose set of Java tools and templates library.
On the other hand, \texttt{OOP
Aufgabe3}\footnote{\url{https://github.com/rwilli/aufgabe3}} aims to manipulate
polygons. \texttt{SimpleShop}\footnote{\url{https://github.com/pase/simpleshop}} is an
electronical shopping system. In addition,
\texttt{Webprot\'{e}g\'{e}}\footnote{\url{https://github.com/protegeproject/webprotege}}
is a collaborative ontology development environment for the Web. Those systems amount to more than 190 KLOC. See Table~\ref{tab:Units} for details in
terms of code lines (LOC), total contract clauses (\#CC) we were able to write
-- following \cite{Estler-etal14} approach, in which the number of contract clauses is a proxy for contract complexity -- as split
into preconditions (\#Pre), postconditions (\#Post), and invariants (\#Inv).\footnote{The clauses
correspond to the contracts we applied in each system.}

%systems table
\begin{table}[ht]
\caption{Case study Systems. LOC shows the code lines (LOC), total contract clauses (\#CC), as split
into preconditions (\#Pre), postconditions (\#Post), and invariants (\#Inv)).}
\label{tab:Units}
\centering
\begin{tabular}{llllll}
\toprule
\bfseries System &  \bfseries LOC & 
\bfseries \#CC &  \bfseries \#Pre &  \bfseries \#Post &
 \bfseries \#Inv \\ \hline
ABC-Music-Player & 1,973 & 115 & 41 & 74 & 0 \\ 
Dishevelled & 110,577 & 2,655 & 1,411 & 1,250 & 0 \\ 
Jenerics & 2,538 & 190 & 105 & 85 & 0 \\ 
OOP Aufgabe3 & 353 & 54 & 28 & 26 & 0 \\
SimpleShop & 472 & 50 & 16 & 15 & 19 \\
Webprot\'{e}g\'{e} & 74,742 & 929 & 351 & 579 & 0 \\ \hline

 \bfseries Total &  \bfseries \totalCode{} &  \bfseries
\totalClauses{} &  \bfseries \totalPre{} &  \bfseries \totalPost{} &
 \bfseries \totalInv{}
\\
\bottomrule
\end{tabular}
\end{table}

%what is precondition, postcondition and invariant in javadoc
%criterio para a traducao
The manual translation abides by the following criteria: method-level comments were considered preconditions
if the comments establish some restriction over the method parameters.
For instance, \texttt{``@param notes - Should not be null and should be of length >= 2''} was
replaced by the following \contractjdoc{}-based expression \texttt{[notes != null \&\& notes.size() >= 2]}, and
postconditions that establish details on the return value of the
methods, e.g. \texttt{``@return Integer the number of edges. Is always >= 3''}
was replaced by \texttt{[@return >= 3]}. Class-level comments make up for
invariants when they describe properties over fields that must be maintained for
all methods of the class.

\subsubsection{Experimental Procedure and Research Method} 

%tasks
%translation
Three researchers applied \contractjdoc{} in \totalSystems{} existing open-source systems
available at GitHub. They followed a bottom-up approach for
writing the \contractjdoc{} contracts: the researchers started applying
\contractjdoc{} in the simplest methods and classes (or interfaces), following
up to the most complex. Contracts followed the Javadoc comments available in
natural language (in English) and some of them were inferred from the methods'
source code.
As result, they wrote \totalClauses{} contract clauses:
\totalPre{} preconditions, \totalPost{} postconditions, and \totalInv{} invariants (see
Table~\ref{tab:Units}).
Figure~\ref{fig:applicationProcess} presents the steps performed by the researchers when applying
\contractjdoc{} to the systems. The process is composed of four steps: 1) generation of the
contracts based on the natural language comments available (as showed in
Section~\ref{sec:systems}); 2) compilation of the contracts by means of
\contractjdocCompiler{} compiler, in order to generate the bytecode enriched
with assertions; 3) the test suite available in each system is run over the
contract-aware bytecode; 4) results of the test suite execution are analyzed and
conformance errors are investigated.

\begin{figure}[h]
\centering
\includegraphics[width=1.0\textwidth]{figs/ContractJDocProcess}
\caption{Steps for applying \contractjdoc{} to Javadoc-annotated systems.}
\label{fig:applicationProcess}
\end{figure}

%contract classification
Concerning the kind of written contracts, we group the contracts according to
the approach of \cite{typeContracts}: application-specific contracts
(AppSpec.) -- the kind of contracts that enforce richer semantic properties;
% implications)
 common-case contracts (Com.Case) -- the kind of contracts that enforce
expected (common) program properties;
% methods do not modify unrelated variables; 
code-repetitive (Repet.) -- the kind of
contracts that repeat exact statements from the code.
% : that a method returns a

%experiment package - link
All systems with the contracts added in this study are available in a replication
package.\footnote{\url{https://goo.gl/yO8or2}; in order to run the
\contractjdoc{} compiler, the folder \textit{aspectjml-lib} must be copied into the folder of each system.}
%when we consider an error
Concerning the verification performed after applying \contractjdoc{} contracts into the systems,
we used the test suites available with the purpose of identifying problems
(four out of \totalSystems{} projects have a test suite available).
Every test case that failed was investigated in order to find out if it was a conformance error in the system.

% applying dbcjdoc
As a secondary goal, the study allowed us to check the expressiveness of \contractjdoc{} and to
evaluate the effort related to adding contracts to existing systems.
In addition, we enhanced the compiler and added features in order to simplify
the process of applying \contractjdoc{} in existing projects.

\subsection{Empirical Study}
\label{sec:experiment}

The goal of the empirical study is to investigate
\contractjdoc{}, for the purpose of evaluation with respect to readability and understandability, from the point of view of
developers in the context of Java programming language. The study presents two factors: the task performed by the
developer (task), and the documenting approach (approach).
% , and the experience level of the developer (experience level)
Those factors have the following treatments: client and supplier\footnote{By client, we mean a class calling the methods provided by an interface, and by supplier, we mean a class implementing the
interface.} -- for task; and Javadoc,
\contractjdoc{}, and JML -- for approach (see Table~\ref{tab:factorsEmpStudy}).

\begin{table}[ht]
\caption{Factors and treatments of the empirical study.}
\label{tab:factorsEmpStudy}
\centering
\begin{tabular}{ll} \toprule
\bfseries Factors & \bfseries Treatments \\
\hline

\multirow{2}{*}{\textbf{Task}} & Client \\
& Supplier \\ \hline 

\multirow{3}{*}{\textbf{Approach}} & \contractjdoc{} \\
 & Javadoc \\
& JML \\ \bottomrule
\end{tabular}
\end{table}

\subsubsection{Participants Recruitment}
\label{sec:expPart}

The subjects who took part in this experiment (called Participants
henceforth) were industry professionals and students from Brazil. Professionals
those who work or already have worked with programming to industry. Students those who have only academic
experience.

The recruitment was done in a convenience sample by the authors: we invited
professionals and students from our network. In total there were 24 participants
in the experiment: 10 industry professionals and 14 students. All participants'
experience includes a good knowledge of Java.


\subsubsection{Study Design}
\label{sec:studyDesign}

We addressed two factors: approach for commenting
source code, task to be performed; with the following treatments: Javadoc, \contractjdoc{}, and JML
-- for approach, and client and supplier -- for the task.
Moreover, we use two Java equivalent interfaces in this experiment: Stack and Queue.
We use a factorial design~\cite{wohlin},  randomly assigning subjects to each combination of the
treatments. For the purpose of this experiment, each triple
$<$approach, task, interface$>$ is called a trial (a combination of treatments).
Since there are three documenting approaches, two tasks and two Java interfaces,
the experiment counts with 12 trials. The assignment Participant --- trial is performed by using a
completely randomized design in order to not bias the results.
% 
% We balance the number of people in each trial because balancing simplifies and strengthens the statistical analysis of data
The experiment uses a balanced design, which means there is the same number of
participants in each trial~\cite{wohlin} -- two in our case.

\subsubsection{Experimental Procedure}
\label{sec:expProcedure}

The experiment was performed offline, i.e., participants received the
experimental material via an online Survey
platform\footnote{An instance
of the platform used is
available online: \url{https://www.formpl.us/form/5671648952844288}} that we use
to collect the results.
% double blind process 
% An example of survey sent to the
% participants can be found online.\footnote{\url{https://goo.gl/forms/ySTsYfKSRcotLayk1}} 
Each
participant received an experiment package, consisting of (i) a statement of consent, (ii) a pretest
questionnaire, (iii) instructions and materials to perform the experiment, and (iv) a post-test
questionnaire. Before the study, we explained to participants what we expected them to do during the
experiment: they were asked to perform an implementation task (a supplier or a client code) for the
provided interface. Each participant received one of the following tasks: create a supplier code for
an interface or a client code for using the interface methods.

Before starting the experiment, we asked each participant to fulfill a
pre-study questionnaire reporting their programming experience (with respect to Java and contract-based programming
experience). After filling in the questionnaire, we randomly selected a task for
each of them.

The first part of the experiment consists
on the following activities: (i) apply a questionnaire pre-experiment -- in order to collect
information on developers experience; (ii) give some kind of training on the documenting approach,
such as JML and \contractjdoc{}; (iii) ask them to execute the tasks -- for each developer will be
given one task for one approach with one Java interface; (iv) apply a post-experiment 
questionnaire -- in order to collect qualitative information about the developers' view of each task.

\subsubsection{Instrumentation}
\label{sec:expInstrumentation}

We performed a pilot with three Java developers in order to fit the
questions structure, the way in which we make the data
available for the developers. As a result, we changed the way of making the
working dataset available to the participants. Initially we were making the documented interface available in a link and the working dataset in another. The
answers to the pilot highlighted this fact and we decided to create a single package containing all
Java classes (all classes needed to the compilation of the code) related to the experiment in a
single URL.

\subsection{Comprehensibility Survey}
\label{sec:survey}

We conducted an exploratory study that involved data collection through a survey with Java
development professionals. 
%This section describes the survey design, participants, results and discussion.
%goal, question, metric
The goal of the survey is to compare three documentation approaches (Javadoc,
\contractjdoc{}, and JML) with respect to comprehensibility, from the point of view of developers. 

\subsubsection{Design}
\label{sec:surveyDes}

%survey method
For this study, we followed a quantitative method based on a web-based survey instrument, suited to
measure opinions and behaviors in response to specific questions~\cite{refSurvey}, in a non
threatening way. 
%web-based survey
% double blind
% The questions were available as an online form\footnote{http://goo.gl/forms/XcEqvPH0Eq920jaA3}. 

%survey structure
The survey
instrument\footnote{\url{https://goo.gl/forms/8W9jUMGCavkkzDj12}} begins with a
purpose of clarification along with a consent term.
Then, a characterization of the
respondent is conducted by some questions related to Java experience and experience with
contract-based programming. Next, the survey is presented: links for three Java interfaces with each
one documented in a different approach is showed, then some questions related to the understanding
of the behavior of a class implementing the interfaces based on the comments available is asked. 

%likert scale
We used Likert-scale questions. In two questions we ask the developers to
choose the most understandable documentation approach: one specific -- related
to the provided interface; and one general, concerning the use of the approach in a general
context.

%pilot study
We also conducted a pilot concerning the
questions and the structure of the programs being used. The pilot consists in asking three Java
developers to test the setup for the survey.
The pilot allowed us to validate the survey's questions and structure.
We did not have to change anything. The developers who participated did not reported issues on the structure that we
used for presenting the needed data for the participation in the study.


\subsubsection{Survey Participants}
\label{sec:surveyPart}

%who to survey - convenience sample
The survey participants are also extracted by means of a 
non-probability sampling technique -- convenience
sampling~\cite{wohlin}: the nearest and most convenient persons are selected as subjects. We send
the survey link to academic and professional mailing lists.
%snowball approach - contacts
In addition, our contacts made a snowball approach, sending the survey to their respective
contact lists, increasing the sample and the number of participants in our study.
%number
The survey was open for three weeks (from June to July 2016) and received 142
answers (from an estimated total of 700 who received the link, 20\% response
rate). From the 142, 51 are professionals and 91 are students.

\subsection{Statistical Methods}
\label{sec:statisticalMethods}

In our experiment, we applied Wilcoxon rank sum test~\cite{statistical} and Kruskal-Wallis rank sum test~\cite{statistical} for comparing the results according to our treatments. For the survey results, we applied Oneway ANOVA test~\cite{statistical}, The Tukey HSD~\cite{statistical} and pairwise comparisons using t tests
with Bonferroni correction~\cite{statistical}; in addition, we applied Wilcoxon rank sum test with continuity correction
tests.

\section{Results}
\label{results}

We start by describing the results for each study in detail, before proceeding to summarize and discuss the observed effects.

\subsection{Experimental Simulation}
\label{sec:expResults}

In Table~\ref{tab:results} we summarize the results from the 24 trials. Those participants mixed graduates working in industry (41.6\%) and graduates during their M.Sc. and Ph.D. studies (58.4\%).



\begin{table}
\centering
\caption{Experimental results, for each treatment (textual Javadoc \emph{JavaDoc}, \contractjdoc{} \emph{ContJDoc} and formal contracts \emph{Formal}. For each API (Queue or Stack) and Task (\emph{Cli} if a client for the API was implemented, \emph{Sup} if an implementation for the API was provided), participants are listed (\emph{Part}) along with the result (\emph{Res}) from our test cases.}
\label{tab:results}
\begin{tabular}{|l|l||l|l||l|l||l|l|} 
\cline{3-8}
\multicolumn{1}{l}{} &              & \multicolumn{2}{l||}{\textbf{JavaDoc }} & \multicolumn{2}{l||}{\textbf{ContJDoc }} & \multicolumn{2}{l|}{\textbf{Formal}}  \\ 
\hline
\uline{DataStr}      & \uline{Task} & Part & Res                              & Part & Res                                & Part & Res                             \\ 
\hline\hline
\uline{Queue}        & \uline{Cli}  & p9   & \greencheck       & p11  & \redcross                                    & p7   & \greencheck                                 \\
                     & \uline{Cli}  & p10  & \greencheck                                 & p12  & \redcross                                    & p8   & \redcross                                \\
                     & \uline{Sup}  & p21  & \greencheck                                 & p23  & \greencheck                                    & p19  &  \greencheck                               \\
                     & \uline{Sup}  & p22  & \redcross                                 & p24  & \greencheck                                    & p20  &  \redcross                               \\ 
\hline\hline
\uline{Stack}        & \uline{Cli}  & p3   & \redcross                                 & p5   & \redcross                                    & p1   & \redcross                                 \\
                     & \uline{Cli}  & p4   & \redcross                                 & p6   & \redcross                                    & p2   & \greencheck                                 \\
                     & \uline{Sup}  & p15  & \greencheck                                 & p17  & \redcross                                    & p13  & \redcross                                \\
                     & \uline{Sup}  & p16  & \greencheck                                 & p18  & \greencheck                                    & p14  & \redcross                                \\
\hline
\end{tabular}
\end{table}



Table~\ref{tab:faults} details the faults that were found in the participants' code.

\begin{table}
\centering
\caption{Reason (fault) for failures in participants' results}
\label{tab:faults}
\begin{adjustbox}{width=\textwidth}
\begin{tabular}{|l|l|l|l|} 
\hline
\multicolumn{4}{|l|}{\textbf{Javadoc} }                                                                                                                                                                         \\ 
\hline
p22                                        & Queue                             & Sup                                & Code error (method \textbackslash{}texttt\{add\})                                         \\ 
\hline
p3                                         & Stack                             & Cli                                & Contract Failure (invariant on all methods)                                               \\ 
\hline
p4                                         & Stack                             & Cli                                & Contract Failure (does not fulfil several post-conditions)                                \\ 
\hline
\multicolumn{4}{|l|}{\textbf{ContractJDoc }}                                                                                                                                                                    \\ 
\hline
p11                                        & Queue                             & Cli                                & Unfinished task                                                                           \\ 
\hline
p12                                        & Queue                             & Cli                                & Unfinished task                                                                           \\ 
\hline
p5                                         & Stack                             & Cli                                & Contract Failure (pre-condition on method \textbackslash{}texttt\{consumeService\})       \\ 
\hline
p6                                         & Stack                             & Cli                                & Code error (internal manipulation)                                                        \\ 
\hline
p17                                        & Stack                             & Sup                                & Contract Failure (invariant unfulfilled in the stack)                                     \\ 
\hline
\multicolumn{4}{|l|}{\textbf{Formal contracts} }                                                                                                                                                                \\ 
\hline
p8                                         & Queue                             & Cli                                & Contract Failure (invariant on all methods)                                               \\ 
\hline
p1                                         & Stack                             & Sup                                & Contract Failure (post-condition on method \textbackslash{}texttt\{removeAccountTop\})    \\ 
\hline
p13                                        & Stack                             & Sup                                & Contract Failure (post-condition unfulfilled on method \textbackslash{}texttt\{pop\})     \\ 
\hline
p14                                        & Stack                             & Sup                                & Contract Failure (post-condition unfulfilled on method \textbackslash{}texttt\{pop\})     \\ 
\hline
p20                                        & Queue                             & Sup                                & Contract Failure (post-condition unfulfilled on method \textbackslash{}texttt\{remove\})  \\ 
\hline\hline
\multicolumn{1}{|c|}{\textbf{Part}} & \multicolumn{1}{c|}{\textbf{API}} & \multicolumn{1}{c|}{\textbf{Task}} & \multicolumn{1}{c|}{\textbf{Type of~Fault}}                                               \\
\hline
\end{tabular}
\end{adjustbox}
\end{table}






% difficulty
Concerning the task performed by the developers, we present in
Figure \ref{fig:tasksEmpirical}
answers on difficulty grouped by the task performed.
The implementation of the documented interface (Supplier task) seemed to be easier
than the task of creating a Client class, however, this difference is not
statistically significant as presented by a Wilcoxon rank sum test~\cite{statistical}
(p-value = 0.07, confidence level of 95\%).
%

\begin{figure*}
\centering
\begin{subfigure}{.3\textwidth}
\includegraphics[width=1\linewidth]{figs/boxplotTasksEmpiricalStudy}
\caption{}
\label{fig:tasksEmpirical}
\end{subfigure}
\begin{subfigure}{.3\textwidth}
\includegraphics[width=1\linewidth]{figs/boxplotApproachesEmpiricalStudy}
\caption{}
\label{fig:approachesEmpirical}
\end{subfigure}
\begin{subfigure}{.3\textwidth}
\includegraphics[width=1\linewidth]{figs/boxplotExperienceEmpiricalStudy}
\caption{}
\label{fig:experienceEmpirical}
\end{subfigure}
\caption{Results of our empirical study with Java developers, on an implementation task based on a
documented-interface, aiming to evaluate the readability and understandability of three approaches
for documenting Java code.}
\label{fig:empiricalResults}
\end{figure*}


% \begin{figure}
% \centering
% \includegraphics[width=1.0\textwidth]{figs/TaskComplexity}
% \caption{Answers grouped by task complexity.}
% \label{fig:taskComplexity}
% \end{figure}

% documenting approach
When grouped by documenting approach (see Figure \ref{fig:approachesEmpirical}),
the Kruskal-Wallis rank sum test showed no difference between the approaches
(p-value = 0.15).

% , the results present Javadoc as the easiest approach to be understood (see Figures \ref{fig:empiricalResults}(c),
% \ref{fig:empiricalResults}(d), and \ref{fig:empiricalResults}(e)). This is
% expected since Javadoc is a well established approach for documenting Java code. In addition, \contractjdoc{} was perceived as easier than JML: there are three answers \textit{Very easy} for \contractjdoc{} whereas
% there is no answer \textit{Very easy} for JML.

% experience
When grouped by experience (Figure \ref{fig:experienceEmpirical}), the Wilcoxon
rank sum test (p-value = 0.45) also does not show differences statistically
significant between professionals and students. 

% source code correct
Javadoc and \contractjdoc{} were the only documenting approaches in
which all participants were able to produce a code satisfying the oracle
(respecting the restrictions available in the comments). On the other hand,
there was one case developed by following the JML documenting approach in which
the contract is not satisfied by the implementation.





\subsection{Judgment Survey}
\label{sec:surveyResults}

142 Java developers answered the survey.
From those, 
51 are professionals (36\%) and 91 are students
(64\%).

%results - in general
With respect to the survey answers, 50.7\% (72) of the Subjects chose Javadoc as
the simplest approach to understand when using it in a general context. In addition,
for 38\% (54) of the subjects Javadoc is also the most understandable approach
with regard to the provided interface.

\begin{figure*}
\centering
\begin{subfigure}{.48\textwidth}
\includegraphics[width=1\linewidth]{figs/boxplotApproachesSurveyStudy}
\caption{All approaches}
\label{fig:allApproaches}
\end{subfigure}
\begin{subfigure}{.48\textwidth}
\includegraphics[width=1\linewidth]{figs/boxPlotJavadocXExperience}
\caption{Javadoc}
\label{fig:javadocExp}
\end{subfigure}
\\[1ex]
\begin{subfigure}{.48\textwidth}
\includegraphics[width=1\linewidth]{figs/boxplotContractJDocXExperience.png}
\caption{\contractjdoc{}}
\label{fig:contractjdocExp}
\end{subfigure}
\begin{subfigure}{.48\textwidth}
\includegraphics[width=1\linewidth]{figs/boxplotJMLXExperience.png}
\caption{JML}
\label{fig:jmlExp}
\end{subfigure}
\caption{Subjects' answers to the individual evaluation of comprehensibility for
each documentation approach. And answers grouped by experience for each
approach.}
\label{fig:surveyResults}
\end{figure*}

% continues
The survey results provided us statistical difference when comparing the the
comprehensibility of the documentation approaches evaluated (see
Figure~\ref{fig:allApproaches}). By performing an Oneway ANOVA test~\cite{statistical} and
a corresponding post hoc analysis we were able to distinguish the three
approaches (p-value $<$ 0.05).
The Tukey HSD~\cite{statistical} and pairwise comparisons using t tests
with Bonferroni correction~\cite{statistical} produced the following p-values:
Javadoc-ContractJDoc = 0.012, JML-ContractJDoc = 0.000, and JML-Javadoc = 0.000.

When analyzing data grouped by experience (Figures~\ref{fig:javadocExp} to
~\ref{fig:jmlExp}) by means of Wilcoxon rank sum test with continuity correction
tests, only for JML we found no statistical difference between Professionals and
Students (p-value = 0.17). For both Javadoc and \contractjdoc{}, Professionals
had perceived the approaches as being easier for comprehensing than Students
(p-value = 0.012 and p-value = 0.004, respectively).


\subsubsection{Comprehensibility Survey}
\label{sec:surveyDiscussion}





\subsection{Case Study}

Table~\ref{tab:caseStudyResults} presents the results of applying \contractjdoc{} to each system.
Column \#Clauses displays the number of clauses manually added in each system.
Column \#Errors presents the number of errors detected by the systems test suite after compiling the source code enhanced with contracts in
\contractjdoc{} approach. Column Time reveals the time (in seconds) needed for compiling the whole
project with its dependencies after applying \contractjdoc{} contracts. Columns \#Com.Case to \#Repet.
show the contract clauses added in each system grouped by type (following the
definitions from ~\cite{typeContracts}).

\begin{table*}[h]
\caption{Case study Results.}
\label{tab:caseStudyResults}
\centering
\begin{tabular}{l l l l l l l l}
\hline
 \bfseries System &
 \bfseries \#Clauses & 
 \bfseries \#Errors & 
 \bfseries Time (s) &
 \bfseries \#Tests &
 \bfseries \#Com.Case &
 \bfseries \#AppSpec. &
 \bfseries \#Repet. \\ \hline
ABC-Music-Player & 115 & 2 & 14 & 30 & 42 & 11 & 62 \\
Dishevelled & 2,655 & 381 & 434 & 2,643 & 1,536 & 151 & 968 \\
Jenerics & 190 & 7 & 20 & 44 & 156 & 0 & 34 \\
OOP Aufgabe3 & 54 & 1 & 4 & 11 & 16 & 30 & 8 \\
SimpleShop & 50 & 0 & 5 & 0 & 30 & 11 & 9 \\
Webprot\'{e}g\'{e} & 930 & 0 & 713 & 0 & 717 & 79 & 133 \\ \hline

 \bfseries Total & 
 \bfseries \totalClauses{} & 
 \bfseries 391 &
 \bfseries 1,185 &
 \bfseries 2,728 &
 \bfseries 2,497 &
 \bfseries 282 &
 \bfseries 1,214
\\
\bottomrule
\end{tabular}
\end{table*}

% Jenerics: 8 - 6 failures, 1 error.

% results for contract types
Concerning the kind of contracts, the only unit in which we wrote more
application-specific contracts was \texttt{OOP Aufgabe3} system (55\% of the
written contracts are application-specific). On the other hand, in \texttt{ABC-Music-Player}, more than 90\% of the contracts remains between common-case and
repetitive code: verifications that strings are not blank, collections are not
empty, or that a method returns a field.
For \texttt{Dishevelled}, the majority of the written contracts is classified as common-case
(57.51\%), other 36.92\% are repetitive with code and only 5.57\% are application-specific.
In addition, all contracts written for \texttt{Jenerics} are related to
verification of nullity from parameters or the return value, thus all contracts
remains between common-case and repetitive code. In \texttt{SimpleShop}, the
written contracts are distributed in the following manner: common-case 60\%, repetitive code 19\%, and application-specific 21\%; again the number of common-case and repetitive code outperforms application-specific contracts. Finally,
in \texttt{WebProt\'{e}g\'{e}}, the distribution is: common-case 77.51\%, repetitive code 14.38\%, and application-specific 8.11\%.  

% conformance errors
When applying \contractjdoc{} to \texttt{ABC-Music-Player}, we found inconsistencies between Javadoc
comments and the source code. The problems occurred in the class \texttt{Utilities} (package
\texttt{sound}) because there are comments concerning a parameter declaring that the value of
this parameter must not be greater than or equal to zero; however in the body of the methods there
is an if-clause that throws exceptions when the value received by the parameter is negative.





\section{Limitations and Threats to validity}
\label{sec:CaseStudyThreats}

External validity refers to generalization. Due to its size, results from the
case study cannot be generalized; its purpose is evaluating applicability and relative usefulness.
%represetantive sample
The sample is not representative, since there is no available estimate of the Javadoc-rich project
population in GitHub, then probability sample is impossible.
Our approach is as systematic as feasible in selecting the evaluated project -- manual translation
does not scale, then the sample contains only \totalSystems{} projects.
Therefore, those systems may not be representative of the real use of Javadoc in real systems; however, we were able
to detect inconsistencies between Javadoc comments and source code, as occurred in
\texttt{Utilities} class (\texttt{ABC-Music-Player} experimental unit) in which the comment for a
parameter of the methods is the right opposite of the expected behavior in the source code.

Another risk is that only 24 developers participated in the experimental study
and 142 in the survey and those samples are not representative for the community
of Java developers. Furthermore, we used only two similar data structure interfaces (queue and stack). In other domains with more complex structures, the results
may vary. In addition, the survey used only one data structure interface:
\texttt{Stack}, for asking about the comprehensibility of the interface behavior.
In other domains with more complex structures, the results can vary considerably.


Internal validity refers to causation: are changes in the dependent variable
necessarily the result of manipulations to treatments? All material for the
empirical study and the survey study is available only in English, therefore,
the experience of the Subject with English can have affected their
comprehensibility of the behavior of the provided interface.

%webprotege much bigger
Construct validity refers to correctly measuring the
dependent variable. \texttt{Dishevelled} and
\texttt{WebProt\'{e}g\'{e}} sizes set them apart from the other systems.
For instance, \texttt{Dishevelled} is more than 56 times bigger than \texttt{ABC-Music-Player}, 43
times bigger than \texttt{Jenerics}, 313 times bigger than \texttt{OOP Aufgabe3}, and 234 times
bigger than \texttt{SimpleShop}.
In order to reduce the threat on the manually-defined contracts,
all systems were annotated and reviewed by three researchers, separately.

The order in which we display the documented
interfaces on the survey form, the questions used for evaluating
comprehensibility, the kind of questions used, and the absence of opened questions
can also threat the construct validity. For dealing with these threats we
perform a pilot before applying the survey and used the results from the pilot
to improve the survey structure. In addition, the answers from developers may not be
representative of their real opinion on difficulty perception; to overcome this
threat we made a space for comments available along with the Likert-scale
questions, which are taken into account when collecting the answers.

\section{Related Work}
\label{sec:relatedWork}

\textbf{Contracts.} 
As discussed, each contract-based approach choo\-ses 
a different trade-off
between expressivity/preciseness, verbosity, freedom, and tooling (e.g., runtime checking).
This is the case of JML~\cite{jml} and Microsoft's Code Contracts~\cite{codeContractsPaper}.
Both enable one to provide full behavioral specifications and their runtime checking.
Nevertheless, they lack support to allow informal specifications or to be available at third-party
library clients~\cite{Parnas2011}. They differ in the way they are written. 
The former is written as Java comments in code, whereas the latter is syntax-based 
and therefore often verbose. Without tool support to extract meaningful specifications
the contracts provided by Code Contracts are even less interesting for third-party 
libraries since they are embedded in C\# programs. Differently from these languages, the contracts expressed in \contractjdoc{} are already embedded in Javadoc comments, which are the standard approach to documenting Java programs and more likely to be available to third-party libraries.

\textbf{Javadoc Comments.}
@TCOMMENT~\cite{atComment} is an approach for testing Javadoc comments, specifically
method properties about null values and related exceptions. The approach consists of two components. The first component takes as input
source files for a Java project and automatically analyzes the English text in Javadoc comments to
infer a set of likely properties for a method in the files. The second component generates random
tests for these methods, checks the inferred properties, and reports inconsistencies. By using
\contractjdoc{}, a developer is able to write contracts richer than those for checking null values and
exceptions (as presented in Section~\ref{sec:caseStudy}).

Zhai et al.~\cite{docAnalysis} present a technique that builds models for Java API functions
by analyzing the documentation. Their models are simpler implementations in Java compared to the
original ones and hence easier to analyze. More importantly, they provide the same functionalities
as the original functions. They argue that API documentation, like Javadoc and .NET documentation,
usually contains wealthy information about the library functions, such as the behavior and exceptions they may throw. Thus it is feasible to generate models for library functions from such API documentation. In this context, the comments in \contractjdoc{} approach can be used as input for the technique in order to improve model generation.

\textbf{Testing.}
Clousot~\cite{clousot} statically checks C\#/Code
Contracts programs. The approach is based on abstract interpretation and analyzes annotated programs
to infer facts (including loop invariants), and it uses this information to discharge proof
obligations. 
AutoTest~\cite{autotest} is a collection of tools that automate the testing process for
Eiffel programs. In AutoTest, contracts are used as oracles to expected outputs for conformance checking of the programs; furthermore, AutoTest uses a randomly-guided tests generation (ARTOO~\cite{artoo}) and
supports mixing manual and automated test. 
\jmloktool{}~\cite{jmlok2} is a tool for dynamically detecting and classifying nonconformances in contract-based programs, applying randomly-generated tests (RGT) for detecting nonconformances, and a heuristics-based approach for nonconformance classification. 
These tools are for contract-based languages; in contrast, we propose and implement an approach for writing contracts in the same language as the source code, such as available in JML and Eiffel, improving documentation.

\textbf{Empirical Studies.}
There are three main related empirical studies about contract usage~\cite{typeContracts,Estler-etal14,Chalin06}.
One common conclusion about is that, in practice, developers use simple and short
contracts~\cite{typeContracts,Estler-etal14}. For instance,
\cite{typeContracts} showed that 75\% of the projects' Code Contracts are checks for the presence of data (e.g., non-null checks).
In our case study (Section~\ref{sec:caseStudy}), almost 93\% (3,711 contract
clauses out of 3,994) of the contracts we wrote remains between checks for the
presence of data and statements repeating the method's return. Chalin studied 84 Eiffel~\cite{eiffel} projects
and pointed out that developers are more likely to use contracts in languages that support them natively, like Eiffel~\cite{eiffel} or Code Contracts~\cite{codeContractsPaper}.
To support both conclusions, to write simple contracts in a native manner, \contractjdoc{} allows a Java programmer
to write those contracts as usual Javadoc comments.

\section{Conclusions}
\label{sec:conclusions}

In this work, we present a new approach for writing comments and an
application of this approach for Java context. The approach
allows the use of Design by Contract~\cite{dbc} in a format closer to
traditional Javadoc comments. When evaluating \contractjdoc{} we found the
approach almost as understandable as traditional Javadoc comments, with the
advantage of being able to check behavior at runtime.
When comparing JML with \contractjdoc{}, the latter features less specification
constructs, although results indicate programmers may feel more comfortable with
it when writing precise behavior for methods. In addition, we found evidence
that \contractjdoc{} is more readable than JML.

% advantages
\contractjdoc{} contracts enable runtime checking by using a language similar to the traditional Javadoc,
and our compiler (\contractjdocCompiler{}) supports new constructs of Java language, such as the
features from the Java 8, such as lambda expressions.
Furthermore, as JML contracts, contracts in \contractjdoc{} may be used in place of defensive
programming, by specifying valid inputs for the methods and shortening the source code.

% future work
As future work, we plan to leverage \contractjdoc{} for
supporting autocomplete when writing the Javadoc comment. For this purpose, we
can use the Jaro-Winkler~\cite{jaro,winkler99} string distance for
autocompleting the comments.

%% Bibliography
\bibliography{ecoop-2019}

\end{document}
